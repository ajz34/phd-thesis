% !TEX root=./chap-04.tex
%-----全局定义-----
\documentclass[type=doctor]{fduthesis}

%-----FDU thesis setup-----
\fdusetup{
    style = {
        font = times,
        cjk-font = fandol,
        font-size = -4,
        fullwidth-stop = mapping,
        % footnote-style = xits,
        hyperlink = color,
        hyperlink-color = default,
        bib-backend = bibtex,
        bib-resource = {../thesis.bib},
        % bib-style = achemso,
        % cite-style = numerical,
        % declaration-page = {declaration.pdf},
        % 插入扫描版的声明页 PDF 文档
        % 默认使用预定义的声明页,但不带签名
        auto-make-cover = false
        % 是否自动生成论文封面(封一)、指导小组成员名单(封二)和声明页(封三)
        % 除非特殊需要(e.g. 不要封面),否则不建议设为 false
    },
    %
    % info 类用于录入论文信息
    info = {
    title = {双杂化密度泛函分子能量与性质\\计算方法的进展与测评},
    title* = {\alert{Thesis Title}},
    % 英文标题
    %
    author = {祝震予},
    supervisor = {徐\quad 昕\quad 教授},
    major = {物理化学},
    degree = academic,
    department = {化学系},
    student-id = {17110220038},
    % date = {2023 年 1 月 1 日},
    % 日期
    % 注释掉表示使用编译日期
    instructors = {
        {徐\quad 昕 \quad 教\quad 授},
    },
    % 指导小组成员
    % 使用英文逗号 “,” 分隔
    % 如有需要,可以用 \quad 手工对齐
    %
    keywords = {\alert{不确定关系, 量子力学, 理论物理}},
    % 中文关键词
    % 使用英文逗号 “,” 分隔
    %
    keywords* = {Uncertainty principle, quantum mechanics, theoretical physics},
    % 英文关键词
    % 使用英文逗号 “,” 分隔
    %
    clc = {O641.12},
    % 中图分类号
    }
}

%-----fduthesis issues-----
% issue #86
\ExplSyntaxOn
\tl_set:Nn \c__fdu_cover_info_align_tl { c @ { \c__fdu_fwid_colon_tl } l }
\ExplSyntaxOff

%-----图表设置-----
\usepackage{siunitx}
\usepackage{enumitem}
\newcommand{\tabnote}[1]{\textsuperscript{\emph{#1}}}

\usepackage{graphicx}
\usepackage{longtable}
\usepackage{longfigure}
\usepackage{subcaption}
\usepackage{float}
\usepackage{lscape}
\usepackage{multicol}
\usepackage{multirow}
\usepackage{arydshln}
\usepackage{dcolumn}
\newcolumntype{d}[1]{D{.}{.}{#1}}
\setlength\dashlinedash{0.5pt}
\setlength\dashlinegap{1.5pt}
\setlength\arrayrulewidth{0.5pt}
\usepackage{rotating}

%-----化学符号-----
\usepackage[version=4]{mhchem}

%-----数学记号----
\newcommand{\bm}{\symbfit}
\allowdisplaybreaks[1]

%-----其它定义-----
\newcommand{\Schrodinger}{Schr\"o\-dinger}
\newcommand{\alert}[1]{{\color{red}{#1}}}

%---------设定区结束----------


\begin{document}

%---------预定设置区----------
\title{\textbf{双杂化密度泛函分子能量与性质计算方法的测评与进展\\第四章草稿}}
\author{祝震予}
\maketitle
\vspace{-10pt}

\tableofcontents

%---------正  文  区----------

\setcounter{section}{3}

\section{双杂化泛函原子体系电子云密度与能量测评}

\subsection{引言}

DFT (密度泛函理论) 是原理上严格的理论;即没有引入任何近似,且必然存在一个对所有体系普适的、严格正确的泛函 $F[\rho]$;它可以导出精确的能量、电子云密度、各种分子或物质性质、以至于精确的波函数本身。

但现实是,我们难以获得严格的密度泛函本身。依 Levy 的设想\cite{Levy-Levy.PNAS.1979},真实泛函需要在完整的波函数空间下作搜索,其代价是巨大的。更晚的讨论表明,所有 $k$-local Hamiltonian 问题 ($k > 2$) 与 $N$ 可表示问题是 QMA 的 (量子计算机下非多项式的、困难的)\cite{Kempe-Regev.SJC.2006, Liu-Verstraete.PRL.2007};进而 DFT 问题也是 QMA 的\cite{Schuch-Verstraete.NP.2009}。具体来说,普适泛函的计算消耗、泛函变量的空间两者总存在其一是 QMA 问题:
\begin{itemize}[nosep]
    \item 使用波函数 $\Psi$ 作为基本变量,而不用密度 $\rho(\bm{r})$ 描述;则由于过大的变量空间,计算消耗是 QMA 的 (Full-CI、QMC 或 Levy 约束搜索);
    \item 电子云密度 $\rho(\bm{r})$ 作为变量,则普适泛函 $F[\rho]$ 的计算消耗是 QMA 的;所有对 $F[\rho]$ 计算消耗将至 P (多项式复杂度) 的尝试都必然是近似 (DFT 理论);
    \item 若不使用单粒子密度 $\rho(\bm{r})$ 而使用双粒子密度 $\rho_2(\bm{r}, \bm{r}')$ 构造泛函,那么即使 $F[\rho_2]$ 的计算复杂度可能降至 P;但为限制双粒子密度 $\rho_2 (\bm{r}, \bm{r}')$ 在其定义域 ($N$ 可表示空间),其代价仍然是 QMA 的 (2-RDM 框架)。
\end{itemize}
因此,任何理论或实现框架都无法简单地解决分子或物质模拟问题。

但从具体实现上,DFT 理论近似的普适泛函 $F[\rho]$ 计算消耗小、且 $\rho(\bm{r})$ 作为变量的空间小,从而广受物质计算工作者的欢迎。而跨越构造普适泛函 $F[\rho]$ 的困难,则成为泛函开发者不懈的追求。

但在如何克服 $F[\rho]$ 构造上的困难,不同的研究者持有不同的态度。一些研究者坚持对 $F[\rho]$ 的形式与性质作深入的研究;若了解愈多 $F[\rho]$ 的严格性质 (exact constraints),就愈接近真实的严格普适泛函\cite{Perdew-Ernzerhof.PRL.1996, Sun-Perdew.PRL.2015, Medvedev-Lyssenko.S.2017}。另一些研究者在承认可预见的未来内,$F[\rho]$ 难以精确地构造的基础上,着重于更准确地描述具体的、当前物质科学所关心的问题;愈多类型的体系与性质可以被精确地描述,就愈接近真实的严格普适泛函\cite{Yu-Truhlar.JCP.2016, Chen-Weinan.JCTC.2020}。

表面上,这两种看法的差异在于构造 $F[\rho]$ 时,是否存在经验参数。一般来说,前者通过严格性质构造 $F[\rho]$ 的途径,会使用较少甚至于没有经验参数。后者则通常对 $F[\rho]$ 作参数化;针对其所关心的物质计算问题,拟定数据训练集与验证集,对参数化的 $F[\rho]$ 作监督学习。而从实践的经验上,对于经验参数拟合的泛函,若计算问题处在训练集或验证集中,则通常有优异的表现;但若超出这些数据集的范围,则其表现很有可能因过拟合现象而粗强人意。相比之下,非经验参数拟合的泛函更有可能在各种物质及其性质上有良好的表现。

目前常用的密度泛函近似中,大多数都带有经验参数;而这些经验参数通常是针对能量性质的数据集而优化的。Medvedev 等\cite{Medvedev-Lyssenko.S.2017}指出,目前绝大多数泛函开发者关注于更精确地描述物质能量;其中不乏声音认为对能量更好的描述,可以给出更精确地泛函。但另一方面,电子云密度 $\rho(\bm{r})$ 作为密度泛函 $F[\rho]$ 的参量,作为连接多电子体系与能量的关键桥梁,却很少被关注到。基于对轻原子与正离子体系的电子云密度系统性的分析与测评,Medvedev 等认为,依泛函提出的年代顺序,2000 年以前发展的泛函在电子云密度上的表现较好、表现出 DFT 理论与其近似的进步;但 2000 年以后发展的泛函中,一部分泛函由于物理上不满足严格的性质、以及过分宽松的多经验参数拟合,使得后来的泛函在电子云密度的总体表现上逐年次第劣化。注意到密度泛函近似的目标是逼近普适泛函;而普适泛函应当能对所有的物质性质作精确的模拟;电子云密度并不例外。因此,对于这部分泛函,Medvedev 无法认同它们正在逼近普适泛函的道路上。同时他们认为,对于非参数或少量经验参数拟合的另一部分泛函,电子云密度确实地随“Jacob 阶梯”\cite{Perdew-Schmidt.ACP.2001}的爬升而愈加精确。



\bibliographystyle{achemso}
\bibliography{../thesis.bib}

\end{document}