%-----字体大小-----
%文档的统一字体还需要手动修改
\newlength{\fntsize}
\fntsize=11pt

%-----全局定义-----
\documentclass[11pt,a4paper,onecolumn]{article}
\usepackage[top=15mm,left=18mm,right=18mm,bottom=20mm]{geometry}
\columnsep=20pt
\usepackage[space]{ctex}
\usepackage{fontspec,xltxtra,xunicode}

%-----字体设置-----
%黑体、楷体使用Adobe系列,宋体、仿宋使用知网华光系列
\setCJKmainfont[BoldFont=AdobeHeitiStd-Regular, ItalicFont=AdobeKaitiStd-Regular]{STSong}
\setCJKsansfont{AdobeHeitiStd-Regular}
%\setmainfont{Minion Pro}
\setCJKfamilyfont{zhhei}{AdobeHeitiStd-Regular}
\providecommand*{\heiti}{\CJKfamily{zhhei}}
\setCJKfamilyfont{zhfs}{FangSong}
\providecommand*{\fangsong}{\CJKfamily{FangSong}}
\setCJKfamilyfont{zhkai}{AdobeKaitiStd-Regular}
\providecommand*{\kaiti}{\CJKfamily{zhkai}}
%\setsansfont[BoldFont=HelveticaNeueLTPro-Md]{HelveticaNeueLTPro-Md}

%-----日期设置-----
\renewcommand{\today}{\kaiti \number\year 年 \number\month 月 \number\day 日}

%-----标题设置-----
\usepackage{titlesec}
\usepackage[toc,page]{appendix}
% \titleformat*{\section}{\Large\sf}
% \titleformat*{\subsection}{\large\sf}

%-----行距设置-----
%1.2代表的是1.5倍行距;设置为2倍行距应打入1.667
\linespread{1.2}

%-----颜色设置-----
\PassOptionsToPackage{svgnames}{xcolor}
\usepackage{xcolor}

%-----书签与引用设置-----
\usepackage[CJKbookmarks]{hyperref}
\hypersetup{%
  colorlinks,
  linkcolor=red,
  filecolor=green,
  urlcolor=blue,
  citecolor=blue,
  pdfstartview=FitH,
  bookmarksnumbered=true}


%-----设置引用-----
\renewenvironment{quotation}
{\list{}{\rightmargin\leftmargin}%
  \item\relax\kaiti\small\hspace{1.7em}}
{\endlist}

%-----首行缩进-----
\usepackage{indentfirst}
\usepackage{setspace}
\setlength{\parindent}{2\fntsize}

%-----图表设置-----
\usepackage{graphicx}
\usepackage[justification=centering]{caption}
\DeclareCaptionFont{heiti}{\heiti}
\DeclareCaptionFont{kaiti}{\kaiti}
\captionsetup{labelfont=bf, textfont=kaiti}
\renewcommand{\figurename}{图}
\newcommand{\reffig}[1]{图~\ref{#1}}
\usepackage{subcaption}
\usepackage{float}
%http://stackoverflow.com/questions/3275622/
%latex-remove-spaces-between-items-in-list
\usepackage{enumitem}

%-----引用设置-----
%参考文献的麻烦的底层修改
%http://blog.sina.com.cn/s/blog_5e16f1770100l3kc.html
\usepackage[square,numbers,super]{natbib}
% \usepackage{natbib}
% \addtolength{\bibsep}{-0.5 em} % 缩小参考文献间的垂直间距
% \renewcommand\bibnumfmt[1]{#1.}  %去掉文末文献列表的[](数字或上标模式)
% \newcommand{\bibnumfont}[1]{\textit{#1}}

%-----关于数学记号的说明与宏包-----
\usepackage{amsmath}
%公式标号
\numberwithin{equation}{section}
%允许分页显示公式
%\allowdisplaybreaks[1]
%环路积分
\usepackage{esint}
%向量与矩阵
%本文中,向量用粗斜体标记,矩阵用粗正体标记
%\bm加粗,\hm重粗
\usepackage{bm}
%花体可以通过\mathscr写出
\usepackage{mathrsfs}
%双线字体可以通过\mathbb调出
\usepackage{amsfonts}
\usepackage{amssymb}
\usepackage{empheq}

% Shorten Equation Spacing
% https://tex.stackexchange.com/questions/69662/how-to-globally-change-the-spacing-around-equations
% https://tex.stackexchange.com/a/69678
\usepackage{xpatch}
\xapptocmd\normalsize{%
  \abovedisplayskip=5pt plus 2pt minus 4pt
  \abovedisplayshortskip=0pt plus 3pt
  \belowdisplayskip=5pt plus 2pt minus 4pt
  \belowdisplayshortskip=4pt plus 2pt minus 2pt
}{}{}

%-----彩色文档框-----
% Beamer Box in Article
%https://tex.stackexchange.com/questions/2504/beamer-blocks-in-ordinary-article-style-document
\usepackage{tcolorbox}
\usepackage{lipsum}
\tcbuselibrary{skins,breakable}
\usetikzlibrary{shadings,shadows}
\tcbset{textmarker/.style={%
    skin=enhancedmiddle jigsaw,breakable,parbox=false,
    boxrule=0mm,leftrule=1.5mm,rightrule=0mm,boxsep=0mm,arc=0mm,outer arc=0mm,
    left=3mm,right=3mm,top=3mm,bottom=3mm,toptitle=1mm,bottomtitle=1mm}}
\newtcolorbox{exercise}{textmarker,colback=ForestGreen!5!white,colframe=ForestGreen!80!white,fontupper=\par,fontlower=\par}
\newtcolorbox{warning}{textmarker,colback=OrangeRed!5!white,colframe=OrangeRed!80!white,fontupper=\fangsong\par,fontlower=\fangsong\par}
\newtcolorbox{hint}{textmarker,colback=Peru!5!white,colframe=Peru!80!white,fontupper=\fangsong\par,fontlower=\fangsong\par}
\newtcolorbox{solution}{textmarker,colback=RoyalBlue!5!white,colframe=RoyalBlue!80!white,fontupper=\kaiti\par,fontlower=\kaiti\par}

\newcommand{\tcbemph}[1]{\noindent\textsf{#1}\par}

%-----化学符号-----
\usepackage[version=4]{mhchem}
% \mhchemoptions{textfontcommand=\sffamily}
% \mhchemoptions{mathfontcommand=\mathsf}

%-----TikZ-----
\usepackage{tikz}
\usetikzlibrary{decorations.pathmorphing}
\usetikzlibrary{decorations.markings}
\usetikzlibrary{arrows.meta}
\usetikzlibrary{quantikz}

%-----其它排版-----
\setlength{\columnsep}{-1.5cm}
\usepackage{multicol}

%---------设定区结束----------