% !TEX root=./chap-05.tex
%-----全局定义-----
\documentclass[type=doctor]{fduthesis}

%-----FDU thesis setup-----
\fdusetup{
    style = {
        font = times,
        cjk-font = fandol,
        font-size = -4,
        fullwidth-stop = mapping,
        % footnote-style = xits,
        hyperlink = color,
        hyperlink-color = default,
        bib-backend = bibtex,
        bib-resource = {../thesis.bib},
        % bib-style = achemso,
        % cite-style = numerical,
        % declaration-page = {declaration.pdf},
        % 插入扫描版的声明页 PDF 文档
        % 默认使用预定义的声明页,但不带签名
        auto-make-cover = false
        % 是否自动生成论文封面(封一)、指导小组成员名单(封二)和声明页(封三)
        % 除非特殊需要(e.g. 不要封面),否则不建议设为 false
    },
    %
    % info 类用于录入论文信息
    info = {
    title = {双杂化密度泛函分子能量与性质\\计算方法的进展与测评},
    title* = {\alert{Thesis Title}},
    % 英文标题
    %
    author = {祝震予},
    supervisor = {徐\quad 昕\quad 教授},
    major = {物理化学},
    degree = academic,
    department = {化学系},
    student-id = {17110220038},
    % date = {2023 年 1 月 1 日},
    % 日期
    % 注释掉表示使用编译日期
    instructors = {
        {徐\quad 昕 \quad 教\quad 授},
    },
    % 指导小组成员
    % 使用英文逗号 “,” 分隔
    % 如有需要,可以用 \quad 手工对齐
    %
    keywords = {\alert{不确定关系, 量子力学, 理论物理}},
    % 中文关键词
    % 使用英文逗号 “,” 分隔
    %
    keywords* = {Uncertainty principle, quantum mechanics, theoretical physics},
    % 英文关键词
    % 使用英文逗号 “,” 分隔
    %
    clc = {O641.12},
    % 中图分类号
    }
}

%-----fduthesis issues-----
% issue #86
\ExplSyntaxOn
\tl_set:Nn \c__fdu_cover_info_align_tl { c @ { \c__fdu_fwid_colon_tl } l }
\ExplSyntaxOff

%-----图表设置-----
\usepackage{siunitx}
\usepackage{enumitem}
\newcommand{\tabnote}[1]{\textsuperscript{\emph{#1}}}

\usepackage{graphicx}
\usepackage{longtable}
\usepackage{longfigure}
\usepackage{subcaption}
\usepackage{float}
\usepackage{lscape}
\usepackage{multicol}
\usepackage{multirow}
\usepackage{arydshln}
\usepackage{dcolumn}
\newcolumntype{d}[1]{D{.}{.}{#1}}
\setlength\dashlinedash{0.5pt}
\setlength\dashlinegap{1.5pt}
\setlength\arrayrulewidth{0.5pt}
\usepackage{rotating}

%-----化学符号-----
\usepackage[version=4]{mhchem}

%-----数学记号----
\newcommand{\bm}{\symbfit}
\allowdisplaybreaks[1]

%-----其它定义-----
\newcommand{\Schrodinger}{Schr\"o\-dinger}
\newcommand{\alert}[1]{{\color{red}{#1}}}

%---------设定区结束----------


\begin{document}

%---------预定设置区----------
\title{\textbf{双杂化密度泛函分子能量与性质计算方法的测评与进展\\第五章草稿}}
\author{祝震予}
\maketitle
\vspace{-10pt}

\tableofcontents

%---------正  文  区----------

\setcounter{section}{4}

\section{高精度基组外推方法在 CCSD(T) 静态极化率计算上的应用}

\subsection{引言}

静态极化率是光学中问题重要的物理量\cite{Marder-Stucky.ACS.1991};它在化学中,也与 Raman 光谱活性\cite{Wilson-Cross.Dover.1955}、有机反应机理\cite{Xing-Pei.HEP.2005}、以及分子间相互作用\cite{Cohen-Tannoudji-Laloe.Wiley.2020}等问题上有概念上的联系。作为计算化学可以导出的物理量,对静态极化率描述的准确与否,也可以用于判断电子结构近似方法精度与有效性。因此,静态极化率是理论与计算化学都十分关心的重要物理量。

相对于静态极化率的概念是动态极化率。动态极化率 $\alpha(\omega)$ 是在特定频率 $\omega$ 下偶极电场扰动下分子能量变化的表征量,而静态极化率是动态极化率的外场频率外推至零的情况即 $\alpha(0)$。本文中,若不作额外说明,极化率一般指静态极化率。

我们希望对双杂化泛函在静态极化率上的表现作评测与研究;为此,我们需要精确的静态极化率参考值。计算方法的终极目标应当是逼近物理真实的结果,因此参考值应当选取这些真实结果的数值;但现实上,真实结果难以获得。为了确定参考值以评测近似计算化学方法的有效性,可能的方案是以精确的实验数据、或高精度的计算数据替代参考值。

若要通过实验数据构成参考值以测评理论计算结果,一个关键问题与困难是如何准确地联系理论结果与实验环境\cite{Mata-Suhm.ACIE.2017}。对于静态极化率问题,我们认为,通过实验数据确定与获得参考值较为困难。这种困难不仅来源于实验本身的精度,也同时来源于难于排除所有实验环境因素;这些因素包括溶剂效应、非谐效应等各种影响,也因静态偶极矩是动态偶极矩在含频电场下频率外推至零的极限、实验上这种外推的矫正难于实现或精确的数据不足。因此,在目前的研究中,电子结构方法的误差经常会被实验本身的误差、以及实验环境与理论模拟之间的差距所产生的误差的总和所掩盖\cite{Hickey-Rowley.JPCA.2014}。因此,相比起使用实验数值作为参照的方式\cite{Hickey-Rowley.JPCA.2014},大多数对静态极化率的测评工作都使用高精度理论计算数值作为参考值\cite{Hammond-Xantheas.JCP.2009, Huzak-Deleuze.JCP.2013, Wu-Thakkar.CPL.2015, Kozlowska-Bartkowiak.PCCP.2019, Hait-Head-Gordon.PCCP.2018, Beizaei-Sauer.JPCA.2021}。

若要使用理论参考值作测评工作,有两点非常关键。一者,为了确保评测相对来说公平且完整,数据集应当要足够大、且足够多样,以使得在测评结果在统计上是有意义的。目前的极化率数据集中,较为接近这一目标的有 Hickey 与 Rowley 设计的 46 分子的数据集 (HR46)\cite{Hickey-Rowley.JPCA.2014},Wu、Kalugina 与 Thakkar 设计的 145 分子数据集 (T145)\cite{Wu-Thakkar.CPL.2015}、以及 Hait 与 Head-Gordon 设计的 132 分子与自由基数据集 (HH132)\cite{Hait-Head-Gordon.PCCP.2018}。二者,作为参考值的电子结构理论计算方法应当足够准确。完全组态相互作用 (Full-CI, Full-\underline{C}onfiguration-\underline{I}nteraction) 结合完备基组 (CBS, \underline{C}omplete \underline{B}asis \underline{S}et) 所给出的结果是最理想的情况,但其计算量巨大,显然是不现实的。HH132 数据集选择使用 CCSD(T)/CBS,即化学中常称为“黄金标准”的 CCSD(T) 方法 (\underline{C}oupled-\underline{C}luster \underline{S}ingles and \underline{D}oubles with perturbative \underline{T}riples)\cite{Cizek-Cizek.Wiley.1969, Raghavachari-Head-Gordon.CPL.1989}、并结合有限基组下的 CBS 外推方法\cite{Nyden-Petersson.JCP.1981, Petersson-Mantzaris.JCP.1988},给出了极化率的参考值。其中,对于较小的体系,用于 CBS 外推的模式是 aCV[Q5]Z\footnote{在本文中,我们约定作为 Dunning 系列基组的 aV$X$Z 代表 aug-cc-pV$XZ$ 基组、aCV$X$Z 代表 aug-cc-pCV$XZ$ 基组 ($X \in \{\mathrm{D, T, Q, 5}\}$)。同时约定,作为双基组 CBS 外推模式,aV[$XY$]Z 代表基组从 aV$X$Z、aV$Y$Z 外推到 aV$\inf$Z 的基组极限近似。Dunning 系列基组具体的 CBS 外推方式将在\alert{后文定义}。};对于较大的体系,用于 CBS 外推的模式是 aCV[TQ]Z\cite{Hait-Head-Gordon.PCCP.2018, Hait-Head-Gordon.JCTC.2018}。由于 CBS 外推所用的基组,几乎已是目前计算化学常用的最大基组。对于 CCSD(T) 在电性质计算问题上的准确性,已有文献对此作较为深入的讨论并给出正面的结论\cite{Halkier-Joergensen.JCP.1999, Monten-Deleuze.MP.2011, Hait-Head-Gordon.JCTC.2018}。因此,HH132 数据集的参考值可以认为是可信的。与此同时,HR46 数据集中理论计算给出参考值的模型是 CCSD/aVTZ,而 T145 数据集是 CCSD(T)/aVTZ;CCSD 被认为在计算电性质时精度较低、而 aVTZ 基组也相对较小,因此这两个数据集的理论计算参考值相对于 HH132 数据集,还有进一步提升的空间。

但需要指出,HH132 之所以可以使用 CCSD(T)/aV[TQ]Z 或 CCSD(T)/aV[Q5]Z 的模型给出参考值,是因为该数据集最大的分子仅包含 3 个非氢原子 (包括 \ce{BHF2}, \ce{ClCN}, \ce{CO2}, \ce{CSO}, \ce{FCN}, \ce{FCO}, \ce{HCCCl}, \ce{HCCF}, \ce{HCNH2}, \ce{HCOOH}, \ce{NaCN}, \ce{NOCl}, \ce{O3}, \ce{OCl2}, \ce{OF2}, \ce{SCl2}, \ce{SF2}, \ce{SO2})、或 7 个原子 (包括 \ce{CH3NH2})。而作为对比,HR46 与 T145 数据集的分子相对来说大许多;HR46 最大的体系包含 15 个原子 (toluene 或 \ce{C7H8});T145 数据集最大的体系包含 14 个原子 (1,4-dithiane 或 \ce{C4H8S2})。HR46 与 T145 数据集中,单个体系含有非氢原子数量最多可达 8 个原子 (例如 6-amino-1H-pyrimidin-2-one 或 \ce{C4H5ON3} 以及 5,5,5-trichloropenta-1,3-diyne 或 \ce{C5HCl3})。因此,HR46 与 T145 数据集难以使用与 HH132 相同的 CCSD(T) 结合大基组 CBS 外推的方式,给出精确地理论计算的参考值。

为了尽可能地推高 HR46 与 T145 数据集的精度,我们考虑到使用组合化学方法。典型的组合化学方法是 G$n$\cite{Pople-Curtiss.JCP.1989, Curtiss-Pople.JCP.1990, Curtiss-Pople.JCP.1991, Curtiss-Pople.JCP.1998, Curtiss-Raghavachari.JCP.2007} 与 W$n$\cite{Martin-Oliveira.JCP.1999, Parthiban-Martin.JCP.2001};其做法是针对特定的电子结构方法 (如 CCSD(T), CCSD, MP2, HF),使用对应的、代价上可以承受的基组进行计算,并最终合理地依计算层级对这些方法-基组组合的结果作线性的加减处理,给出更为精确地计算结果。FPA (\underline{F}ocal-\underline{P}oint \underline{A}nalysis) 与组合化学方法有着类似的思想\cite{East-Allen.JCP.1993}。由于能量或极化率张量等性质具有可加性,因此可以将总量上占大头的 HF 方法下大基组计算的结果、MP2 与 HF 方法差减部分在中等基组下计算结果、以及计算复杂度最高但总量上较小的 CCSD(T) 与 MP2 方法差减部分在较小基组下的计算结果作加和。这样的结果一般总是比单纯使用低级别的 HF 结合大基组、或高级别的 CCSD(T) 结合小基组要更为准确。FPA 已经应用于计算化学的各种问题,包括反应生成热\cite{East-Allen.JCP.1993, Nielsen-Schaefer.JCP.1997}、构象能\cite{Csaszar-Schaefer.JCP.1998, Tschumper-Tschumper.JCP.2001, Kahn-Kahn.JCC.2008}、非共价相互作用\cite{Tschumper-Quack.JCP.2002, Jurecka-Hobza.PCCP.2006, Marshall-Sherrill.JCP.2011}、激发能\cite{Bokareva-Godunov.IJQC.2008}、核磁共振屏蔽常数\cite{Sun-Xu.JCP.2013, Wang-Xu.JCP.2018}、以及本工作所关心的静态极化率问题\cite{Huzak-Deleuze.JCP.2013, Monten-Deleuze.MP.2011}。

本工作中,我们将着眼于提升 HR46 与 T145 静态极化率数据集的质量至 CCSD(T)/CBS 的精度。为此,我们首先在可以承受 aCV[Q5]Z 级别计算的小分子体系下,对 HF, MP2, CCSD 与 CCSD(T) 作详细的静态极化率基组收敛性分析。这部分工作的目标是确认对于静态极化率问题,FPA 确实可以给出与 HH132 数据集参考值相似精度的结果,即 FPA 是有效的。随后,我们将 FPA 应用于 HR46 与 T145 数据集的参考值计算上;这些参考值的精度将是 CCSD(T)/aCV[Q5]Z 级别的。我们希望这些高精度的参考值能为未来化学工作者在极化率测评,特别是对密度泛函方法的测评上,提供有效的数据来源。

\subsection{具体方法与实现细节}

\subsubsection{误差测评标准}

对于一个数据量为 $N$ 的数据集的参考值是 $\tilde r = \{ r_n \}$,若近似计算方法的结果是 $\tilde c = \{ c_n \}$、以及作为分母的数据 $\tilde d = \{ d_n \}$,那么数据 $\tilde c$ 的相对方均根误差 (RelRMSD, \underline{Rel}ative \underline{R}oot \underline{M}ean \underline{S}quared \underline{D}eviation) 定义为
\begin{equation}
    \text{RelRMSD} (\tilde c, \tilde r, \tilde d) = \sqrt{\frac{1}{N} \sum_{n = 1}^N \left( \frac{c_n - r_n}{d_n} \right)^2} \times 100\%
\end{equation}
不少情况下,参考值 $\tilde r$ 与分母值 $\tilde d$ 是相同的;此情形下将简记 $\text{RelRMSD} (\tilde c, \tilde r, \tilde d)$ 为 $\text{RelRMSD} (\tilde c, \tilde r)$。一般来说,更低的 RelRMSD 数值,意味着近似计算方法的表现统计上来看越精确。

RelRMSD 误差多大,才认为是可以被接受或容许的范围?首先,本工作的目标是对较大分子体系的极化率作逼近 CBS 外推模型的 CCSD(T)/aCV[Q5]Z 精度的计算;因此,该级别精度在本工作中视为最精确的参考值 $\tilde r$。由于静态极化率的基组极限误差与实验误差通常在 0.5\% 以内\cite{Brakestad-Frediani.JCTC.2020, Rumble-Rumble.CRC.2021};因此我们认为,对于稍低等级的 FPA 模型或稍小基组下的计算结果,如果其相比于 CCSD(T)/aCV[Q5]Z 的 RelRMSD 误差在 0.5\% 以下,即可认为是足够准确的。

% 关于实验误差,我暂时没能找到 102 版 CRC 常数手册的具体说明。在 2010 年的 90 版 Atomic and Molecular Polarizabilities Table 8 (pdf p. 1648--1649 或 10-200--10-201),注脚中提到了 0.5% 的误差,但也同时指出了实验条件上是含频率的情况、以及数据年代问题。

\subsubsection{电子结构方法与软件}

本工作的极化率同时使用数值与解析的方式计算;其计算方式定义在\alert{下几小节}作定义。MP2 与 HF 解析极化率是通过 \textsc{PySCF} (commit 3d592b0)\cite{Sun-Chan.WCMS.2018, Sun-Chan.JCP.2020}的扩展程序脚本 \textsc{dh} (ver 0.1.3)\cite{dh-0.1.ajz34} 所实现。正因为该程序可以高效地实现包括双杂化泛函在内的 MP2 型相关能解析极化率,因此本工作中涉及到的所有分子的 MP2 极化率,在引入 RI 近似的前提下,都可以以最精确的 CBS 外推模式 aCV[Q5]Z 实现。对于 CCSD 与 CCSD(T) 方法,本工作使用 \textsc{Q-Chem} (ver 5.1.1)\cite{Shao-Head-Gordon.MP.2015, Kaliman-Krylov.JCC.2017}作数值极化率计算。这部分计算将不引入 RI 近似。本工作不对 post-HF 计算过程引入冻核近似 (FC, \underline{F}rozen \underline{C}ore)。自由基与其他开壳层体系的 HF 自洽场参考态通过自旋非限制性计算给出。

本工作将在\alert{后文}对极化率的数值与解析之间的误差、RI 近似所导致的误差、以及基组误差进行分析。最终汇报的极化率数值中,HF 与 MP2 是通过 RI 近似下解析给出 (这两个方法将分别记为 RI-JK 与 RI-MP2);CCSD 与 CCSD(T) 的结果将是 RI-MP2 解析极化率、以及更高阶贡献的数值极化率的加和。为更清晰明了地表示极化率数值的构成,本文将使用简记记号;这些记号列于表 \ref{tab.5.1}。该表格的所有记号是针对单个分子或自由基体系的;对于一个数据集合,我们将上标波浪号。举例而言,$\text{RelRMSD} (\Delta \tilde \alpha_{\textsf{(2)}/\text{aVTZ}}, \Delta \tilde \alpha_{\textsf{(2)}/\text{ref}}, \tilde \alpha_{\textsf{CCSD(T)}/\text{ref}})$ 代表 MP2/aVTZ 同性极化率相对于高精度 MP2 参考值的相对方均根误差;该相对误差计算所用的分母,是高精度 CCSD(T) 参考值。

\begin{table}[ht]
    \centering
    \caption{本工作中使用的电子结构方法及其对应极化率的记号\tabnote{a}}
    \label{tab.5.1}
    \begin{tabular}{ll}
    \hline
    记号 & 说明 \\
    \hline
    num & 数值极化率 \\
    anal & 解析极化率 \\
    conv & 传统电子积分方法,即无 RI 近似的计算方法 \\
    $\alpha$ & 同性极化率 \\
    $\gamma$ & 异性极化率 \\
    $\Delta \alpha_\textsf{(2)}$ & $\alpha^\text{anal}_\textsf{RI-MP2} - \alpha^\text{anal}_\textsf{RI-HF}$ \\
    $\Delta \alpha_\textsf{D}$ & $\alpha^\text{num}_\textsf{CCSD} - \alpha^\text{num}_\textsf{MP2}$ (不引入 RI 近似) \\
    $\Delta \alpha_\textsf{(T)}$ & $\alpha^\text{num}_\textsf{CCSD(T)} - \alpha^\text{num}_\textsf{CCSD}$ (不引入 RI 近似) \\
    $\Delta \alpha_\textsf{D(T)}$ & $\alpha^\text{num}_\textsf{CCSD(T)} - \alpha^\text{num}_\textsf{MP2}$ (不引入 RI 近似) \\
    $\Delta \alpha_\textsf{HF}$ & $\alpha^\text{anal}_\textsf{RI-HF}$ \\
    $\Delta \alpha_\textsf{MP2}$ & $\alpha^\text{anal}_\textsf{RI-MP2} = \alpha_\textsf{HF} + \Delta \alpha_\textsf{(2)}$ \\
    $\alpha_\textsf{CCSD}$ & $\alpha^\text{num}_\textsf{CCSD} - \alpha^\text{num}_\textsf{MP2} + \alpha^\text{anal}_\textsf{RI-MP2} = \alpha_\textsf{HF} + \Delta \alpha_\textsf{(2)} + \Delta \alpha_\textsf{D}$ \\
    $\alpha_\textsf{CCSD(T)}$ & $\alpha^\text{num}_\textsf{CCSD(T)} - \alpha^\text{num}_\textsf{MP2} + \alpha^\text{anal}_\textsf{RI-MP2} = \alpha_\textsf{HF} + \Delta \alpha_\textsf{(2)} + \Delta \alpha_\textsf{D(T)}$ \\
    \hline
    \end{tabular}
  
    \raggedright
    \par\tabnote{a} 上述所有应用于同性极化率 $\alpha$ 的记号,将同样应用于异性极化率 $\gamma$。
\end{table}

\subsubsection{数据集}

本工作着重考察的数据集是 HR46 与 T144。原始的 HR46 数据集\cite{Hickey-Rowley.JPCA.2014}包含 46 个体系 (化合物结构参考附录图 \ref{fig.fig-s1});其最精确的计算模型是在 MP2/aVTZ 结构下 CCSD/aVTZ 极化率,且该工作中使用了冻核近似。该数据集的特点是
\begin{itemize}[nosep]
    \item 包含的原子种类较多,包括 H, C, N, O, F, S, P, Cl, Br 等元素;
    \item 在 15 个原子以内的限制下,包含大多数常见有机官能团与无机共价键或离域键;
    \item 分子或自由基的选取契合化学反应、污染控制、生物化学与能源化学的需求;
    \item 除了 3 个体系是自由基体系 (\ce{^2NO}, \ce{^3O2}, \ce{^3SO},上标的数字表示自旋多重度),其余体系均为非自旋极化的体系 (non-spin-polarized)。
\end{itemize}
在本工作中,HR46 中的所有分子的几何结构经过 RI-MP2/aVTZ 级别作优化,且没有启用冻核近似\alert{盲审通过后将结构的链接链上}。所有极化率的计算也在此几何结构下实现。

原始的 T145 数据集\cite{Wu-Thakkar.CPL.2015}包含 145 个有机体系。该数据集的分子使用 B3LYP/aVTZ 下的优化结构,最精确的极化率计算模型是 CCSD(T)/aVTZ。T145 数据集是 TABS 数据集\cite{Blair-Thakkar.CTC.2014}的子集。TABS 数据集本身是一大类着眼于药物化学、有机化学与生物化学应用的、自旋非极化的数据集。TABS 数据集的分子数量为 1641;一方面该数据集本身数量较大、另一方面为使该数据集用于机器学习,Blair 等人对 TABS 挑选 298 分子作为极化率与分子容量的训练集为 T298 数据集\cite{Blair-Thakkar.CPL.2014}。但由于并非所有 T298 数据集中的分子都可以承受 CCSD(T)/aVTZ 的计算代价,因此 Wu 等人从 T298 选出 145 个较小的分子,以给出极化率参考值更精确的数据集,即 T145 数据集\cite{Wu-Thakkar.CPL.2015}。本工作中,我们选取其中的 144 个分子,并称其为 T144 数据集。相比于 T145 数据集,我们排除了第 1363 号分子 (TABS 分子编号,分子名称为 2,3-dihydro-1,3-oxazole);这是因为 TABS 数据集提供的分子构型中,1363 号分子与分子式对应的结构相差两个氢原子,即可能存在数据的损坏。T144 数据集的分子构型使用 TABS 原始数据库提供的 B3LYP/aVTZ 几何结构\cite{Blair-Thakkar.CTC.2014} (化合物结构参考附录图 \ref{fig.fig-s2-1})。

关于 HR46 与 T144 数据集化合物名称的变化与勘定,参考附录 \ref{sec.T145-HR46-name-change}。

本工作的一个关键点是基组收敛性评测。为作有效的评测,测评的体系应可承受高精度大基组 (如 CCSD(T)/aCV5Z 级别模型) 的计算量与资源需求。在 HR46 与 T144 数据集中,14 个体系 (\ce{Cl2}, \ce{CO}, \ce{CO2}, \ce{H2O}, \ce{N2}, \ce{NH3}, \ce{^3O2}, \ce{PH3}, \ce{SH2}, \ce{SiH4}, \ce{^3SO}, \ce{SO2}, \ce{FCN}, \ce{HCHS}) 可以承受大计算量的计算。这 14 个数据也同样出现在 HH132 数据集中。在测评基组收敛性问题时,这 14 个小体系的分子构型将采用 NIST 计算化学数据库中实验的数据\cite{NIST.CCCBDB};这些构型也同样为 HH132 数据集所采用。

\subsubsection{基组与 RI 近似}

在本工作中,我们将使用 Dunning 系列基组 aV$X$Z 与 aCV$X$Z ($X = \mathrm{D, T, Q, 5}$)\cite{Dunning-Dunning.JCP.1989, Kendall-Harrison.JCP.1992, Woon-Dunning.JCP.1993, Peterson-Dunning.JCP.1994, Wilson-Dunning.JMST.1996, VanMourik-Dunning.MP.1999, Wilson-Dunning.JCP.1999, VanMourik-Dunning.IJQC.2000, Peterson-Dunning.JCP.2002, Hill-Peterson.JCP.2010}。H 与 Br 原子并不出现在 aCV$X$Z 系列基组中;对于这两个原子作 aCV$X$Z 级别计算时,将不加说明地替换为对应的 aV$X$Z 基组。RI 近似可以大幅减少 MP2 的计算耗时,并且不会产生严重的误差损耗,是性价比相当高的技术手段\cite{Vahtras-Feyereisen.CPL.1993}。本工作的解析极化率计算中,我们对 Weigend 算法实现的 RI-JK\cite{Weigend-Weigend.PCCP.2002} 与 RI-MP2 相关贡献\cite{Weigend-Haettig.JCP.2002}计算均使用等比辅助基 (ETB, \underline{E}ven-\underline{T}empered auxiliary \underline{B}asis set)\cite{Stoychev-Neese.JCTC.2017};等比参数设为 \textsc{PySCF} 程序默认的 $\beta = 2$。由于基于 \textsc{PySCF} 的扩展程序 \textsc{dh} 的高效率实现、以及对大基组下内存的合理控制,使得本工作中所有体系都可以在 MP2/aCV5Z 下完成计算。作为自洽场的 RI-JK 方法的能量收敛精度为 $10^{-10}$ Hartree。

\subsubsection{数值极化率}

在\alert{第三章的讨论}中,我们已经了解极化率是对外电场强度的二阶梯度;依外电场 $\pmb{\mathcal{E}}$ 在三个空间方向取向的大小 $(\mathcal{E}_x, \mathcal{E}_y, \mathcal{E}_z)$,极化率张量 $\bm{\alpha}$ 是 $3 \times 3$ 的对称矩阵:
\begin{equation}
    \alpha_{ts} = - \frac{\partial^2 E}{\partial \mathcal{E}_t \partial \mathcal{E}_s} \quad t, s \in \{ x, y, z \}
\end{equation}
作为物理上可观测的量,本工作着重测评同性极化率 $\alpha$\footnote{通过标量记号表示,用以区分作为矩阵的极化率张量 $\bm{\alpha}$。}和异性极化率 $\gamma$:
\begin{align}
    \alpha &= \frac{1}{3} \left( \alpha_{xx} + \alpha_{yy} + \alpha_zz \right) = \mathrm{tr} (\bm{\alpha}) \\
    \gamma &= \frac{1}{\sqrt{2}} \left( (\alpha_{xx} - \alpha_{yy})^2 + (\alpha_{yy} - \alpha_{zz})^2 + (\alpha_{zz} - \alpha_{xx})^2 + 6 (\alpha_{xy}^2 + \alpha_{yz}^2 + \alpha_{zx}^2) \right)^{1/2}
\end{align}
对于数值极化率的求取过程,对角元部分 $\alpha_{tt}$ ($t \in \{ x, y, z \}$) 通过线性的三点格式作数值导数给出:
\begin{equation}
    \alpha_{tt} = - \frac{\partial^2 E}{\partial \mathcal{E}_t^2} = - \frac{1}{h^2} \left( E(- \bm{h}_t) - 2 E(\bm{0}) + E(\bm{h}_t) \right) + o(h^2)
\end{equation}
而非对角元部分 $\alpha_{ts}$ ($t, s \in \{ x, y, z \}, \, t \neq s$) 则通过平面上的三点格式作数值导数给出:
\begin{equation}
    \alpha_{ts} = - \frac{\partial^2 E}{\partial \mathcal{E}_t \partial \mathcal{E}_s} = - \frac{1}{4h^2} \left( E(- \bm{h}_t - \bm{h}_s) - E(- \bm{h}_t + \bm{h}_s) - E(\bm{h}_t - \bm{h}_s) + E(\bm{h}_t + \bm{h}_s) \right) + o(h^2)
\end{equation}
其中,$\bm{h}_t, \bm{h}_s$ 分别是 $t, s$ 方向外加微扰电场强度矢量:
\begin{equation*}
    \bm{h}_x = (h, 0, 0)^\dagger, \quad \bm{h}_y = (0, h, 0)^\dagger, \quad \bm{h}_z = (0, 0, h)^\dagger
\end{equation*}
标量 $h$ 是微扰电场的强度;在本工作中,我们对所有体系均使用 $h = 0.004$ au。关于使用该外加微扰强度的合理性,我们在\alert{SI}中作讨论。

\newpage

\subsection{附录}

\subsubsection{HR46 与 T144 数据集分子结构}

\begin{figure}[H]
    \centering
    \caption{HR46 数据集分子结构}
    \label{fig.fig-s1}
    \includegraphics[width=0.85\textwidth]{assets/fig-s1.pdf}
\end{figure}

\begin{figure}[H]
    \centering
    \caption{T144 数据集分子结构}
    \label{fig.fig-s2-1}
    \includegraphics[width=0.85\textwidth]{assets/fig-s2-1.pdf}
\end{figure}

\setcounter{figure}{\value{figure} - 1}

\begin{figure}[H]
    \centering
    \caption{(续图)}
    \label{fig.fig-s2-2}
    \includegraphics[width=0.85\textwidth]{assets/fig-s2-2.pdf}
\end{figure}

\setcounter{figure}{\value{figure} - 1}

\begin{figure}[H]
    \centering
    \caption{(续图)}
    \label{fig.fig-s2-3}
    \includegraphics[width=0.85\textwidth]{assets/fig-s2-3.pdf}
\end{figure}

\setcounter{figure}{\value{figure} - 1}

\begin{figure}[H]
    \centering
    \caption{(续图)}
    \label{fig.fig-s2-4}
    \includegraphics[width=0.85\textwidth]{assets/fig-s2-4.pdf}
\end{figure}

\setcounter{figure}{\value{figure} - 1}

\begin{figure}[H]
    \centering
    \caption{(续图)}
    \label{fig.fig-s2-5}
    \includegraphics[width=0.85\textwidth]{assets/fig-s2-5.pdf}
\end{figure}

\newpage

\subsubsection{体系名称的更改}
\label{sec.T145-HR46-name-change}

本工作会对原先 T144 与 HR46 数据集的化合物名称作一定更改。我们尽量依照 IUPAC 推荐命名规则 (PIN, \underline{P}referred \underline{I}UPAC \underline{N}ame) 进行命名\cite{Favre-Powell.RSC.2013}。举例而言,“acetone”(丙酮) 将被命名为“propane-2-one”(丙-2-酮)。这类名称更改是平凡的。

HR46 原文献\cite{Hickey-Rowley.JPCA.2014}中,正文中名称为“thymine”的分子应命名为“cytosine”(胞嘧啶,或 PIN 命名 6-amino-1\textit{H}-pyrimidin-2-one)。其补充信息中的命名是正确的。

T144 数据集中,我们依据 TABS 数据库所提供的分子坐标文件,对下述 4 个分子的名称作勘定,并示于表 \ref{tab.T144-name-errata}。

\begin{table}[H]
    \centering
    \caption{T144 数据集名称勘定}
    \label{tab.T144-name-errata}
    \begin{tabular}{lll}
    \hline
    ID   & Original   Name                & Updated   Name                    \\ \hline
    0129 & carbonochloridodithioic   acid & methyl   carbonochloridodithioate \\
    0399 & hydroxymethylperoxide          & hydroperoxymethanol               \\
    1193 & 2-bromoethanone                & 2-bromoethen-1-one                \\
    1630 & 5-methyl-2\textit{H}-tetrazole          & 5-methyl-1\textit{H}-tetrazole             \\ \hline
    \end{tabular}
\end{table}

\subsubsection{}

\newpage

\bibliographystyle{achemso}
\bibliography{../thesis.bib}

\end{document}