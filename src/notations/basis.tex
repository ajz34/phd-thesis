% !TEX root=./notations.tex

\chapter{原子轨道基组与程序}

\section{原子轨道基组}
\label{sec.A.basis-set}

下述表格列出本工作所使用的基组、基组基数、基组简称,以及其对应的参考文献。

部分基组的不同元素是由不同参考文献所定义。这里所列举的参考文献,仅针对本工作所涉及的原子。

基组 daug-def2-universal-jkfit 是针对 def2-universal-jkfit 基组,使用基于 Woon 与 Dunning 所提出的方法\cite{Woon-Dunning.JCP.1994},在各角量子数上增加两个弥散基函数;增加弥散基函数所用的程序是 \textsc{Basis Set Exchange} 的 \textsc{Python} API。

本工作还使用到自动生成的 ETB 辅助基\cite{Stoychev-Neese.JCTC.2017}。

下表中,Dunning 系列基组的 $X = \mathrm{D, T, Q, 5}$,对应的基组基数分别是 $\zeta = 2, 3, 4, 5$;Jensen 系列基组的 $n = 1, 2, 3, 4$,对应的基组基数分别是 $\zeta = 2, 3, 4, 5$。正文中,基组 aug-cc-pV$X$Z 常简写为 aV$X$Z,aug-cc-pCV$X$Z 常简写为 aCV$X$Z,aug-pc-$n$ 常简写为 apc$n$。

\begingroup
\setlength{\LTleft}{-20cm plus -1fill}
\setlength{\LTright}{\LTleft}

\begin{longtable}{ll}
    \toprule
    基组名称 & 参考文献 \\ \midrule
    \endhead
    \bottomrule
    \endfoot
    %
    \multicolumn{2}{l}{\textbf{Dunning} (correlation-consistent)} \\
    cc-pV5Z & \citenum{Dunning-Dunning.JCP.1989} \\
    cc-pV5Z-jkfit & \citenum{Weigend-Weigend.PCCP.2002} \\
    cc-pV5Z-rifit & \citenum{Haettig-Haettig.PCCP.2005} \\
    ccecp-cc-pVQZ & \citenum{Bennett-Mitas.JCP.2017} \\
    aug-cc-pωCV$X$Z & \citenum{Dunning-Dunning.JCP.1989, Kendall-Harrison.JCP.1992, Peterson-Dunning.JCP.2002} \\
    aug-cc-pV$X$Z & \citenum{Dunning-Dunning.JCP.1989, Kendall-Harrison.JCP.1992, Woon-Dunning.JCP.1993} \\
    aug-cc-pV$X$Z-jkfit & \citenum{Weigend-Weigend.PCCP.2002} \\
    aug-cc-pV$X$Z-rifit & \citenum{Weigend-Haettig.JCP.2002, Haettig-Haettig.PCCP.2005} \\
    aug-cc-pCV$X$Z & \citenum{Dunning-Dunning.JCP.1989, Kendall-Harrison.JCP.1992, Peterson-Dunning.JCP.2002, Woon-Dunning.JCP.1993} \\
    %
    \midrule
    \multicolumn{2}{l}{\textbf{Karlsruhe}} \\
    def2-TZVPP & \citenum{Metz-Dolg.JCP.2000, Peterson-Dolg.JCP.2003, Weigend-Ahlrichs.PCCP.2005} \\
    def2-QZVPP & \citenum{Metz-Dolg.JCP.2000, Peterson-Dolg.JCP.2003, Weigend-Ahlrichs.PCCP.2005, Weigend-Ahlrichs.JCP.2003} \\
    def2-QZVPPD & \citenum{Metz-Dolg.JCP.2000, Peterson-Dolg.JCP.2003, Weigend-Ahlrichs.PCCP.2005, Weigend-Ahlrichs.JCP.2003, Rappoport-Furche.JCP.2010} \\
    def2-universal-jkfit & \citenum{Weigend-Weigend.JCC.2008} \\
    daug-def2-universal-jkfit & \citenum{Weigend-Weigend.JCC.2008, Woon-Dunning.JCP.1994} \\
    def2-TZVPP-rifit & \citenum{Hellweg-Klopper.TCA.2007, Haettig-Haettig.PCCP.2005, Weigend-Ahlrichs.CPL.1998} \\
    def2-QZVPP-rifit & \citenum{Hellweg-Klopper.TCA.2007, Haettig-Haettig.PCCP.2005} \\
    def2-QZVPPD-rifit & \citenum{Hellweg-Rappoport.PCCP.2015, Hellweg-Klopper.TCA.2007, Haettig-Haettig.PCCP.2005, Weigend-Ahlrichs.CPL.1998} \\
    %
    \midrule
    \multicolumn{2}{l}{\textbf{Jensen} (polarization-consistent)} \\
    aug-pc-$n$ & \citenum{Jensen-Jensen.JCP.2001, Jensen-Jensen.JCP.2002, Jensen-Jensen.JCP.2002a, Jensen-Helgaker.JCP.2004, Jensen-Jensen.JPCA.2007} \\
\end{longtable}

\endgroup

\section{软件与程序}

本工作所使用到的重要程序列表如下:

\begingroup
\setlength{\LTleft}{-20cm plus -1fill}
\setlength{\LTright}{\LTleft}

\begin{longtable}{ll}
    \toprule
    软件或程序 & 参考文献 \\
    \midrule
    \textsc{PySCF} & \citenum{Sun-Chan.WCMS.2018, Sun-Chan.JCP.2020} \\
    \textsc{Libcint} & \citenum{Sun-Sun.JCC.2015} \\
    \textsc{Gaussian 09} & \citenum{Gaussian09} \\
    \textsc{Gaussian 16} & \citenum{Gaussian16} \\
    \textsc{LibXC} & \citenum{Lehtola-Marques.S.2018} \\
    \textsc{Basis Set Exchange} & \citenum{Feller-Feller.JCC.1996, Schuchardt-Windus.JCIM.2007, Pritchard-Windus.JCIM.2019} \\
    \textsc{Q-Chem} & \citenum{Shao-Head-Gordon.MP.2015} \\
    \textsc{PyQMC} & \citenum{Wheeler-Wagner.JCP.2023} \\
    \textsc{dh} ($< 0.1.9$) & \citenum{dh-0.1.ajz34} \\
    \textsc{dh} ($\geqslant 0.1.9$) & \citenum{dh.ajz34} \\
    \bottomrule
\end{longtable}

上述程序中,部分有必要引用的关键性功能列举如下:

\begingroup
\setlength{\LTleft}{-20cm plus -1fill}
\setlength{\LTright}{\LTleft}

\begin{longtable}{lll}
    \toprule
    软件或程序 & 关键性功能 & 参考文献 \\
    \midrule
    \textsc{PySCF} & ETB 自动辅助基生成 & \citenum{Stoychev-Neese.JCTC.2017} \\
    \textsc{Q-Chem} & CCSD(T) 能量计算 & \citenum{Kaliman-Krylov.JCC.2017} \\
    \bottomrule
\end{longtable}

未录入本工作,但在课题早期发展过程中使用、对课题推进有较大帮助的程序与关键性功能列举如下:

\begingroup
\setlength{\LTleft}{-20cm plus -1fill}
\setlength{\LTright}{\LTleft}

\begin{longtable}{lll}
    \toprule
    软件或程序 & 关键性功能 & 参考文献 \\
    \midrule
    \textsc{NWChem} & & \citenum{Valiev-Jong.CPC.2010} \\
    \textsc{NWChem} & xDH 一阶梯度 & \citenum{Su-Xu.JCC.2013} \\
    \textsc{Gaussian 09} & xDH 能量 & \citenum{Zhang-Goddard.PNAS.2009} \\
    \textsc{Gaussian 09} & xDH 二阶梯度 & \citenum{Gu-Xu.JCTC.2021} \\
    \textsc{XCFun} & DFT 格点函数四阶梯度 & \citenum{Ekstroem-Ruud.JCTC.2010} \\
    \textsc{Multiwfn} & & \citenum{Lu-Chen.JCC.2012} \\
    \bottomrule
\end{longtable}

    
\endgroup