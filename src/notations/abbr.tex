% !TEX root=./notations.tex

\chapter{术语表}

\section*{波函数理论与近似}

\begingroup
\setlength{\LTleft}{-20cm plus -1fill}
\setlength{\LTright}{\LTleft}

\begin{longtable}{lll}
    \toprule 简称 & 中文术语 & 英文术语 \\ \midrule \endhead
    \bottomrule \endlastfoot
    %
    \multicolumn{3}{l}{\textsf{波函数理论}} \\
    & 第一性 & \emph{ab initio} \\
    WFT & 波函数理论 & wavefunction theory \\
    SCF & 自洽场 & self-consistent field \\
    post-HF & (专指波函数理论的) 后自洽场 & post Hartree-Fock \\
    PT & 微扰 & perturbation \\
    CI & 组态相互作用 & configuration interaction \\
    Full-CI & 完全组态相互作用 & full configuration interaction \\
    CC & 偶合簇 & coupled-cluster \\
    CEPA & 耦合电子对近似 & coupled electron pair approximation \\
    CPF & 耦合电子对泛函 & coupled pair functional \\
    IEPA & 独立电子对近似 & independent electron pair approximation \\
    %
    \midrule
    \multicolumn{3}{l}{\textsf{波函数组成与性质}} \\
    OO & 轨道优化 & orbital-optimized \\
    SS & 自旋相同 & same-spin \\
    OS & 自旋相反 & opposite-spin \\
    SCS & 自旋组分缩放 & spin-component-scaled \\
    & 正交变换不变性 & unitary invariance \\
    & 大小可延展性 & size extensivity \\
    & 正则自洽场 & canonical SCF \\
    & (非) 限制性方法 & (un)restricted \\
    & 闭壳层 & closed-shell \\
    & 开壳层 & open-shell \\
    RDM & 约化密度矩阵 & reduced density matrix \\
    & 弛豫密度矩阵 & response/relax/Lagragian density matrix \\
    %
    \midrule
    \multicolumn{3}{l}{\textsf{其他第一性方法}} \\
    RPA & 无规相近似 & random phase approximation \\
    QMC & 量子蒙特卡洛 & quantum Monte Carlo \\
    DMC & 扩散蒙特卡洛 & diffusion Monte Carlo \\
    VMC & 变分蒙特卡洛 & variational Monte Carlo \\
\end{longtable}

\endgroup
    
\section*{密度泛函理论与近似}

\begingroup
\setlength{\LTleft}{-20cm plus -1fill}
\setlength{\LTright}{\LTleft}

\begin{longtable}{lll}
    \toprule 简称 & 中文术语 & 英文术语 \\ \midrule \endhead
    \bottomrule \endlastfoot
    %
    \multicolumn{3}{l}{\textsf{密度泛函理论}} \\
    DFT & 密度泛函理论 & density functional theory \\
    DFA & 密度泛函近似 & density functional approximation \\
    post-SCF & 后自洽场 & post self-consistent field \\
    & 无 (电子) 相互作用体系 & noninteracting system \\
    & $N$ 可表示性 & $N$-representability \\
    AC & 绝热路径 & adiabatic connection \\
    GLPT2 & 二阶 G{\"o}rling-Levy 微扰 & 2\textsuperscript{nd}-order G{\"o}rling-Levy perturbation \\
    OEP & 有效优化势 & optimized effective potential \\
    %
    \midrule
    \multicolumn{3}{l}{\textsf{密度泛函近似类型}} \\
    LDA & 局域密度近似 & local density apprioximation \\
    LSDA & 局域密度近似 & local spin-density apprioximation \\
    GGA & 广义梯度近似 & generalized gradient approximation \\
    meta-GGA & 广义梯度的梯度近似 & meta-generalized gradient approximation \\
    hyb & 杂化 (泛函) & hybrid (functional) \\
    RSH & 长短程分离杂化 (泛函) & range-separate hybrid (functional) \\
    & 局域混合 (泛函) & local hybrid (functional) \\
    DH & 双杂化 (泛函) & doubly hybrid (functional) \\
    xDH & XYG3 型双杂化 (泛函) & XYG3-type doubly hybrid (functional) \\
    bDH & B2PLYP 型双杂化 (泛函) & B2PLYP-type doubly hybrid (functional) \\
    DSD & 弥散矫正的 SCS DH & dispersion corrected SCS DH \\
    \midrule
    \multicolumn{3}{l}{\textsf{密度泛函组成与性质}} \\
    LR & 长程效应 & long-range \\
    SR & 短程效应 & short-range \\
    NL & 离域 & non-local \\
    & 半定域 & semi-local \\
    SP & 自旋极化的 & spin-polarized \\
    NSP & 自旋非极化的 & non-spin-polarized \\
\end{longtable}

\endgroup
    
\section*{程序与技术术语}

\begingroup
\setlength{\LTleft}{-20cm plus -1fill}
\setlength{\LTright}{\LTleft}

\begin{longtable}{lll}
    \toprule 简称 & 中文术语 & 英文术语 \\ \midrule \endhead
    \bottomrule \endlastfoot
    %
    \multicolumn{3}{l}{\textsf{电子积分及其近似}} \\
    & 重叠积分 & overlap integral \\
    & (无其他外场的) 单电子积分 & Hamiltonian core integral \\
    ERI & 电子互斥积分 (双电子积分) & electron-repulsion integral \\
    $n$c-$m$e & $n$ 中性、$m$ 电子积分 & $n$-center $m$-electron integral \\
    conv ERI & 传统 4c-2e ERI & conventional ERI \\
    J 积分 & 库伦积分 & Coulomb integral \\
    K 积分 & 交换积分 & exchange integral \\
    & 规范原点 & gauge origin \\
    GIAO & 规范不变原子轨道 & gauge-invariant atomic orbital \\
    RI & 恒等算符 (简化近似) & resolution-of-identity (approximation) \\
    THC & 张量超分解 & tensor hyper-contraction \\
    %
    \midrule
    \multicolumn{3}{l}{\textsf{基组与基组外推}} \\
    & 辅助基组 & auxiliary basis set \\
    FPA & & focal-point analysis \\
    CBS & 完备基组 (极限) & complete basis set (limit) \\
    ETB & & even-tempered basis set \\
    ECP & 有效内层电子势 (赝势) & effective core potential \\
    %
    \midrule
    \multicolumn{3}{l}{\textsf{误差量标}} \\
    MAD/MAE & 平均绝对值误差 & mean absolute deviation/error \\
    RMSD/RMSE & 方均根误差 & root mean squared deviation/error \\ 
    WTMAD-2 & (GMTKN55) 第二型加权 MAD & weighted MAD (scheme 2) \\
    MEPUB & 平均键绝对值误差 & mean unsigned error per bond \\
    RDF & 径向分布函数 & radial distribution function \\
    MNAE & 归一后的绝对值误差 & median normalized absolute error \\
    %
    \midrule
    \multicolumn{3}{l}{\textsf{计算机术语}} \\
    & 元素乘法 & elementwise multiplication \\
    FMA & 乘积累加运算 & fused-multiply-add \\
    FLOPs & 浮点运算数 & floating-point operations \\
    FLOPS & 每秒浮点运算效率 & floating-point operations per second \\
    DRAM & 动态随机访问内存 & dynamic random access memory \\
    & (CPU 的物理) 核心 & (physical) core (of CPU) \\
    & 线程 & thread \\
    NUMA nodes & 非一致内存访问节点 & non-uniform memory access nodes \\
    I/O & 输入输出 & in/out \\
    & 实际程序运行时间 & wall/elapsed time \\
    QMA & 量子计算机下非多项式的 & quantum Merlin-Arthur \\
    %
    \midrule
    \multicolumn{3}{l}{\textsf{电子结构梯度理论}} \\
    CP & 耦合微扰 & coupled-perturbed \\
    CP-SCF & 耦合微扰自洽场 & coupled-perturbed SCF \\
    & 骨架导数 & skeleton/core derivative \\
\end{longtable}

\endgroup
    
\section*{化学概念术语}

\begingroup
\setlength{\LTleft}{-20cm plus -1fill}
\setlength{\LTright}{\LTleft}

\begin{longtable}{lll}
    \hline 简称 & 中文术语 & 英文术语 \\ \hline \endhead
    \hline \endfoot
    %
    HOMO & 最高占据分子轨道 & highest occupied molecular orbital \\
    LUMO & 最低非占分子轨道 & lowest unoccupied molecular orbital \\
    HOMO/LUMO gap & HOMO 与 LUMO 能级差 & energy gap between HOMO and LUMO \\
    IR & 红外光谱 & infrared spectroscopy \\
    UV-Vis & 紫外可见光光谱 & ultra-violet and visable spectroscopy \\
    Raman & Raman 光谱 & Raman spectroscopy \\
    NMR & 核磁共振谱 & nuclear magnetic resonance \\
    MS & 质谱 & mass spectrometry \\
    XRD & X 射线衍射谱 & X-ray diffraction \\
    STM & 扫描隧道显微镜 & scanning tunneling microscopy \\
    & 原位红外 & \emph{in situ} IR \\
    SERS & 表面增强拉曼光谱 & surface-enhanced Raman spectroscopy \\
    SEM & 扫描电化学显微镜 & scanning electron microscopy \\
\end{longtable}

\endgroup