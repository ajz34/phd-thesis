% !TEX root=./dummy-intro.tex

\chapter{后记}
% \addcontentsline{toc}{chapter}{\normalfont \sffamily 后记}

\section*{论文工作相关}

这里将对正文中部分工作的背景、不足与展望作展开。作为正文的补充,这里并不打算作充分与细致的讨论,但希望可以对后续工作与其他研究者有一定启发。

\subsection*{第 2 章相关工作相关}

在第 \alertref{sec.2.title} 章中,我们使用了 MP2/cr 方法,针对部分具有挑战性的过渡金属与静态相关计算问题上,提出了 XYG6+1 模型,并对目前流行的 MP2 型双杂化泛函作改进和测评。

这里的工作只是往前尝试走的一小步。在我不非常充分的认知里,2005 年前后,当 B3LYP、PBE0 等经典近似,被广泛验证并认为对普通有机分子的能量与性质有实用的定量描述时\cite{Koch-Holthausen.Wiley.2001},计算化学方法开发者们认为可以进一步地尝试更有挑战性的问题。其中两个重要的问题,是弱相互作用与强相互作用问题。

弱相互作用问题通常也称为动态相关问题 (部分讨论该问题的综述如 \citenum{Cremer-Cremer.MP.2001, Cohen-Handy.MP.2001, Handy-Cohen.MP.2001})。由于波函数理论中,MP2 方法通常认为可以有效处理动态相关;而 MP2 相关能在密度泛函中的引入确实有效地解决了弱相互作用计算问题。因此,尽管在 2005 年前后,弱相互作用仍然是具有挑战性的问题;但从现在的眼光来看,该问题至少有了定量的解决方案。

当然,这样的解决方案是否是最优的,这个问题仍然是值得争辩的。MP2 的计算量尽管对于小体系确实可以接受,但对于原子数过百的体系仍然会是负担 (特别是内存开销上)。以低阶泛函结合 DFT-D\cite{Grimme-Krieg.JCP.2010, Smith-Sherrill.JPCL.2016, Caldeweyher-Grimme.JCP.2019}、MBD\cite{Tkatchenko-Scheffler.PRL.2009, Tkatchenko-Scheffler.PRL.2012}、VV10\cite{Vydrov-VanVoorhis.JCP.2010} 为代表的计算量较低的弥散矫正方法也对部分弱相互作用问题有良好的表现;但到目前为止,在以 GMTKN55 为代表的数据集上,这些模型仍然没有在 WTMAD-2 等测评标准下有比 XYG7 为代表的双杂化泛函更为良好的表现。因此,传统的低阶密度泛函以及弥散矫正方法,仍然有发展的空间和需求。


