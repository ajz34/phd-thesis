% !TEX root=./dummy-intro.tex

\chapter{后记}
% \addcontentsline{toc}{chapter}{\normalfont \sffamily 后记}

\section*{论文工作相关}

这里将对正文中一些工作的背景、不足与展望作展开。作为正文的补充,这里并不打算作充分与细致的讨论,但希望可以对后续工作与其他研究者有一定启发。

\subsection*{第 2 章工作相关:作为泛函近似方法}

在第 \alertref{sec.2.title} 章中,我们使用了 MP2/cr 方法,针对部分具有挑战性的过渡金属与静态相关计算问题上,提出了 XYG6+1 模型,并对目前流行的 MP2 型双杂化泛函作改进和测评。这里的工作只是往前尝试走的一小步。

在我的认知里,2005 年前后,当 B3LYP、PBE0 等经典近似,被广泛验证并认为对普通有机分子的能量与性质有实用的定量描述时\cite{Koch-Holthausen.Wiley.2001},计算化学方法开发者们认为可以进一步地尝试更有挑战性的问题。其中两个重要的问题,是弱相互作用与强相互作用问题。

弱相互作用问题通常也称为动态相关问题 (部分讨论该问题的综述如 \citenum{Cremer-Cremer.MP.2001, Cohen-Handy.MP.2001, Handy-Cohen.MP.2001})。由于波函数理论中,MP2 方法通常认为可以有效处理动态相关;而 MP2 相关能在密度泛函中的引入确实有效地解决了弱相互作用计算问题。因此,尽管在 2005 年前后,弱相互作用仍然是具有挑战性的问题;但从现在的眼光来看,该问题至少有了定量的解决方案。

当然,这样的解决方案是否是最优的,这个问题仍然是值得争辩的。MP2 的计算量尽管对于小体系确实可以接受,但对于原子数过百的体系仍然会是负担 (特别是内存开销上)。以低阶泛函结合 DFT-D\cite{Grimme-Krieg.JCP.2010, Smith-Sherrill.JPCL.2016, Caldeweyher-Grimme.JCP.2019}、MBD\cite{Tkatchenko-Scheffler.PRL.2009, Tkatchenko-Scheffler.PRL.2012}、VV10\cite{Vydrov-VanVoorhis.JCP.2010} 为代表的计算量较低的弥散矫正方法也对部分弱相互作用问题有良好的表现 (尽管或许 VV10 在体系较大时才能体现相对较小的计算代价);这类方法近年仍然有长足的进步,涌现出不少出色的工作\cite{Kirkpatrick-Cohen.S.2021, Liu-He.NCS.2022}。但到目前为止,在以 GMTKN55 为代表的数据集上,这些模型仍然在 WTMAD-2 等测评标准下尚未达到小于 \SI{3}{kcal.mol^{-1}} 的误差。因此,传统的低阶密度泛函以及弥散矫正方法,仍然有发展的空间和需求。

但强相互作用问题,有时也称为静态相关问题\cite{Cremer-Cremer.MP.2001}、左右相关问题\cite{Handy-Cohen.MP.2001, Cohen-Handy.MP.2001},或化归为分数电荷与自旋问题\cite{Cohen-Yang.S.2008, Cohen-Yang.JCP.2008, Mori-Sanchez-Yang.PRL.2009},迄今为止尽管有许多模型上的尝试与理论框架的搭建\cite{Hesselmann-Goerling.PRL.2011, Lan-Yanai.JCP.2013, Mezei-Kallay.JCTC.2015, Zhang-Scheffler.NJP.2016, Goerling-Goerling.PRB.2019, Shee-Head-Gordon.JPCL.2021, Zhang-Xu.JPCL.2021, Kirkpatrick-Cohen.S.2021, Santra-Martin.JPCL.2022};但就我局限的认知,目前还没有一类完美地满足下述所有条件的方法:
\begin{enumerate}[nosep]
    \item 在主要测评集上有良好表现、GMTKN55 数据集的 WTMAD-2 误差小于 \SI{3}{kcal.mol^{-1}};
    \item 在分子解离问题上,以闭壳层的波函数限制下,不仅可以正确描述近解离趋势、同时在分子解离到无限远处也能给出定量上正确的趋势;
    \item 不直接引入多参考方法,且方法是黑箱的、不需要用户手动指定诸如活性轨道等参数;若要引入多参考信息,其计算复杂度不能是活性轨道数的阶乘或幂次;
    \item 满足正交不变性,不具有自相互作用误差;
    \item 计算量上不显著超过 RI-MP2、且不引入轨道优化 (即不能直接是 OO-MP2 或 CC2 的衍生方法);
    \item 可以实现梯度性质、并有良好的测评表现。
\end{enumerate}
除此之外,大小一致性必须得到满足。这里 Monte Carlo 方法也算作多参考方法。

以本工作的 XYG6+1/cr 为例。上述第 1 点尽管比较乐观,但目前还未完整地经受 GMTKN55 以外数据集下完整的测评;因此,对于将 XYG6+1 模型下的泛函近似,将其扩展到更广泛的强关联或静态相关问题,需要非常谨慎。第 2 点就算是对普通双杂化或低阶泛函有少许改进,但显然不算达成了目标。第 4 点上正交不变性也有少许缺陷;自相互作用误差不可能严格为零,但由于 $E_\textmt{x}^\textmt{exact}$ 在 XYG6+1/cr 中占比为相当大的 0.85、而 MP2/cr 与 MP2 作为相关能也都不具有自相互误差,因此预期自相互作用误差较小。第 6 点尽管也相对乐观,但 MP2 衍生方法经常不满足正交不变性,而这又进一步地会遇到轨道简并情况下梯度数值不稳定的问题 (即附录 \alertref{sec.3.non-canonical-mp2-gradient} 的结论与程序实现策略很可能无法适用于 MP2/cr 或 IEPA 等方法;数值梯度也可能会遇到些许问题)。

除了上述的困难,准确参考值数据的获得也存在一些困难。多参考方法被认为可以很好地解决静态相关问题,但同时其动态相关的描述存在困难、或者程序实现存在一定的不稳定性。多参考方法本身的计算复杂度已经很高,在其之上的微扰理论的计算代价更是一般有机化学体系难以接受的\cite{Finley-Serrano-Andres.CPL.1998, Angeli-Malrieu.JCP.2001, Angeli-Malrieu.CPL.2001, Malmqvist-Gagliardi.JCP.2008, Pulay-Pulay.IJQC.2011}。即使是相当高精度的多参考方法,也可能对一些简单的模型体系无法很好地正常描述\cite{Vancoillie-Veryazov.JCTC.2016}。在计算设备沿着 Moore 定律发展的今天,也有科研工作者看到多参考方法应用于实际问题的希望,在降低计算量、拓展多参考方法应用场景上\cite{Ren-Shuai.WCMS.2022, Li-Chen.PRR.2022, Xiang-Li.JCTC.2024},或对多参考方法引入动态相关的方式上作改进\cite{Mitra-Gagliardi.JCTC.2023, Feng-Xu.chemRxiv.2023},有不少出色的工作。

密度泛函理论与近似的发展与流行,始终与其计算量较低有关;双杂化泛函之所以有潜力成为未来计算化学应用的主力,我相信很大程度上也是因为 Moore 定律下计算容量的发展、以及存在高效的近似计算策略\cite{Almloef-Almloef.CPL.1991, Weigend-Ahlrichs.CPL.1998, Hohenstein-Martinez.JCP.2012, Parrish-Sherrill.JCP.2012}。对于低阶泛函,作为自洽场方法额外地,其梯度与激发态性质容易实现,因而可以在较少的计算量、以及与 Hartree-Fock 相似的程序框架下,实现大量的固体物质、精细光谱模拟 (这或许也是双杂化泛函作为非变分泛函的缺点、以及其亟待解决的问题)。因此,从我的视角出发,若多参考方法还不能在不超过 100 原子的 3-$\zeta$ 或 4-$\zeta$ 基组下,能与低阶或双杂化密度泛函近似有同一水平的计算代价;那么在当下,在密度泛函中引入多参考方法、并将其成规模地应用于实际问题的时机可能不太成熟。但从另一个角度看,Kohn-Sham 定理在能级并不真的简并的前提下,仍然是严格的;而 Kohn-Sham 方程必然可以导出单参考波函数也能描述的电子云密度。因此,如何在 Kohn-Sham 框架或与之接近的框架下,有效解决静态相关问题,还是值得研究的问题。尽管相信很困难,但如果这个思路上能有所突破,那么就有希望在较少的计算量下解决静态相关问题。不论解决这一问题的最终路径,是通过第一性理论推演、还是大数据学习,更快的、更高通量的高精度多参考方法数据总是有帮助的。因此,我也很期待多参考方法的长足发展。

\subsection*{第 2 章工作相关:作为低阶泛函矫正的方法}

对于 xDH 型双杂化泛函,其理论的根源是 G{\"o}rling-Levy 二阶微扰理论;这点已经广为接受与承认,但似乎并不是那么直观。作为一种现在或许不太提及的视角\cite{Zhang-Xu.CC.2010},xDH 型泛函也可以看作是一种对 B3LYP 的矫正方案;这个视角下,或许应当认为 xDH 方法与 X1 神经网络方法\cite{Wu-Xu.JCP.2007}是非常类似的:xDH 方法成功与否,决定于其是否能弥补 B3LYP 等低阶泛函本身的不足。在绪论与正文中,我们已经表明 xDH 确实在诸多反应能与能垒和性质计算上的表现提升,这是 xDH 型泛函成功的地方。相反的问题是,xDH 方法如果存在困难或者局限,到底其来源是什么?

对于能量计算,xDH 型泛函确实存在一些局限;对于 HOMO/LUMO gap 较小体系存在的问题,已经在第 \alertref{sec.2.title} 章中有所展开。最直观的发现是 MP2 在 HOMO/LUMO gap 较小时有严重的负误差。我们也确实作了一些尝试,以避免 xDH 产生过于严重的误差。无论读者如何看待目前 XYG6+1 模型的结果,我想至少可以说,我们在成对电子能引入策略上有所尝试,并有比较系统的研究和测评。

但这个局限是否应归因为 MP2 型相关能的 xDH 没有成功弥补 B3LYP 作为低阶泛函本身的不足吗?我认为是,但不仅如此。如果从解离曲线的形貌出发 (以图 \alertref{fig.2.curve-N2} 为例),我们可以说 XYG6+1 模型下的泛函近似取得了一些进展,但它应该认为是 B3LYP 严重的正误差、与 XYG6+1 相对于 B3LYP 矫正部分的严重的负误差的相互抵消。误差抵消在计算化学中确实是常见的情况、使用得当的时也会产生漂亮的矫正方案;但就当前的例子而言,我觉得很难说这还是一种漂亮的做法:正负误差不管是原理上、还是测评结果上,都不见得保证定性上处在相同的数量级上。从这个视角出发,成对电子方法对 MP2 严重负误差的改进,只是用来更好地平衡 B3LYP 严重的正误差;如果确实将 xDH 视作一种低阶泛函矫正方法,那么被矫正的对象 (B3LYP、PBE0 等) 也必须要是足够好的胚;如果 B3LYP 等低阶泛函的严重正误差无法有效地避免,基于此参考态的 xDH 型泛函潜在的局限就算能缓解、也或许无法根除。因此,如果我们仍然要基于单参考态下作相关能矫正,或许不仅需要针对矫正部分的计算策略作改进\cite{Zhang-Xu.JPCL.2019},同时也需要保证低阶泛函或参考态本身有良好的描述\cite{Becke-Becke.TCC.2014, Ai-Su.JPCL.2021, Kirkpatrick-Cohen.S.2021, Liu-He.NCS.2022}。我相信这其中还有可以挖掘的空间;但也受能力所限、也因为视野始终局限在很少的测评体系以及很少的计算方法上,此刻我还没有寻觅到那一口清泉。

本论文的工作完全是基于单参考态的。但目前对多参考态的计算,其对静态相关有很好的描述、但动态相关仍然存在一些困难。多参考下的密度泛函的定义与计算代价是相对容易的\cite{LiManni-Gagliardi.JCTC.2014, Feng-Xu.chemRxiv.2023},但多参考微扰理论\cite{Finley-Serrano-Andres.CPL.1998, Angeli-Malrieu.JCP.2001, Angeli-Malrieu.CPL.2001, Malmqvist-Gagliardi.JCP.2008, Pulay-Pulay.IJQC.2011}会比单参考下的微扰理论困难许多、同时计算量也非常大。简单的微扰、或不同于普通密度泛函的、但同时计算代价较低的能量矫正策略,或许有希望有效地修正多参考方法对静态相关描述的困难。

\subsection*{附录 A 工作相关}

附录 \alertref{sec.3.title} 中,我们实现了 RI 近似下 DH,特别是 xDH 的静态极化率计算,并尽可能地将程序框架做到可以扩展其他相关形式的泛函。

这是就读期间耗费最多精力的课题;尽管如此,到目前为止,相对于其他科研工作者、或已经发表的工作,这份工作还没有新的突破;不管是功能、易用与易扩展程度、甚至是效率上,都还称不上是新的突破。

之所以会想到踏入计算化学的领域,我想 Dirac 的话语起到过相当程度的影响\cite{Dirac-Dirac.PRSLA.1929}:
\begin{quotation}
    \it
    The underlying physical laws necessary for the mathematical theory of a large part of physics and the whole of chemistry are thus completely known, and the difficulty is only that the exact application of these laws leads to equations much too complicated to be soluble.
\end{quotation}
这句话写在了各种教科书中,相信这句话也影响过许多人;不论是学生,或是编写教科书的前辈们。尽管现在想来,这些话语背后,还意味着新的物理与化学现象在当时尚未发现;但如果只是关注物质的运动方程与状态函数,那么如何有效地数值模拟,便是最重要的技术问题。回想到这篇论文的工作,回想到附录 \alertref{sec.3.title} 冗长的文段;我总觉得希望讲些什么,希望把一些东西讲清楚;但这样的冲动显然不应有我潜意识里所期望的价值,因为若要问它带来了什么,为计算化学理论带来了哪些新的认识和简化,为大众对化学现象的认识带来了哪些提升,我发现给不出令人满意的说辞。起笔论文的时候,我没有注意到编写当时还是正文内容的附录 \alertref{sec.3.title} 的动机;但最终,我认为附录 \alertref{sec.3.title} 的内容是想要表达,就算没有成果,曾经在这个问题上也有普通人的努力。像这样仅仅是宣泄情绪的文段,或许不应出现在论文中;但不忍删去,遂将其从论文起笔时的正文拉入最终的附录。

技术的进步确实是重要的。最近几年,以深度网络为代表的机器学习与人工智能方法盛行;其大背景或许就是 Moore 定律发展到了计算能力与潜力快速超过基于演绎的知识的时代,以至于人们可以依赖于具体计算过程不那么具象的参数化模型。我不清楚是否这两件事是有关联的:计算化学非常深地依赖于张量运算;而机器学习的兴起将张量运算带入了新的高度,理论上与效率上兼有。后者或许意外地有潜力推动所有以张量计算为基础的学科。算子融合、缓存局域优化、Roofline 等等概念,已经渐渐地成为机器学习效率优化的常见技术或关心的问题。作为高性能计算的传统课题,高效的矩阵乘法库函数实现也越来越便利\cite{vanZee-vandeGeijn.ATMS.2015}、以至于高效的任意张量乘法的实现也成为可能\cite{Matthews-Matthews.SJSC.2018}。将来的计算化学程序或许将会逐渐重视这些或受机器学习或高性能计算发展而带动的工程经验。

同时,开源社区的兴起也对计算化学程序有相当大的改观。化学领域有一些开源项目,以高效库函数的方式实现了底层的算法与功能\cite{Valeyev.libint, Lehtola-Marques.S.2018, Sun-Sun.JCC.2015},解放了许多计算化学工作者,让他们更加专注于顶层的算法设计。但同时,中层涉及到张量运算的部分,大多数程序仍然将其作为关键的业务逻辑实现;但实际上,这类张量运算程序,对于功能不同的软件,通常又有相当程度的共通部分、或者说有不少比较同质化的部分。我不清楚这会不会是不切实际的愿望,但未来或许会有一种类似于 LAPACK 的标准,在计算化学程序上更进一步地拆分业务逻辑与需要高效实现的中层算法,以更大程度上地解放方法开发者在程序设计上的开发效率。

\textsc{PySCF} 提供了一种相当有意思的思路与具体实现,它将业务逻辑关键部分以 \textsc{Python} 实现、而拆分出效率关键部分以 C 实现\cite{Sun-Chan.WCMS.2018, Sun-Chan.JCP.2020};基于此,本工作的项目得以使用纯 \textsc{Python},也可以给出优于其他计算化学程序的效率。我本人非常感激有 \textsc{PySCF} 与 \textsc{Psi4} 等项目的存在:尽管机器学习中有 \textsc{Tensorflow} 与 \textsc{Pytorch} 的巨大成功,使得 \textsc{Python/C} 混合高效编程成为必然可以实现的工程实践;但真的存在开源项目将其成功实现,并不见得是历史的必然。但在具体工程实现上,我自己也确实遇到过基于 \textsc{Python} 的程序存在效率瓶颈的问题;这也使得即使当前工作的 RI-MP2 极化率就算效率比其他程序高,也未必比数值梯度快的重要缺陷。这导致本工作程序的唯一特色是,解析梯度的数值精度与稳定性比数值梯度高。一定程度上,也是因为考虑到此,我认为附录 \alertref{sec.3.title} 的工作应当看作是未完成的;需要承认,这受能力所限、也受精力所限。

除此之外,现在已经有众多的异于 RI-MP2 的低计算复杂度加速算法,包括但不限于 Laplace-Transform\cite{Almloef-Almloef.CPL.1991}、ISDF\cite{Dong-Lin.JCTC.2018, Qin-Yang.JPCA.2020}、Local RI\cite{Ihrig-Blum.NJP.2015}、DLPNO\cite{Riplinger-Neese.JCP.2013, Pinski-Neese.JCP.2015}。这些算法在梯度性质上是否也有效率上显著的突破、这些算法的提升极限是多少、如何实现这些算法、如何在广泛的 DH 框架下应用这些算法,也是这份工作所没能完成的遗憾、以及将来有必要解决的课题。

在完成本论文期间、以及论文初稿后的一段时间里,我也尝试了单节点 CPU 下闭壳层 RI-MP2 解析静态极化率程序效率优化的工作。目前已经有一些比较积极的初步结果:在较小的体系下,程序 \textsc{dh} 的效率还能有大约 2--3 倍提升;在较大的体系下 (内存无法容纳 $n_\mathrm{occ}^2 n_\mathrm{vir}^2$ 个双精度浮点数),我们可以避免直接在内存或硬盘储存全部的 $t_{ij}^{ab}$ 张量,从而避免耗时较大的硬盘交互过程,对 CPU 计算能力的利用率在 60\% 以上,并相对于 \textsc{dh} 有 8 倍左右效率提升。尽管必须承认,如果我们对实现算法的理解是正确的,闭壳层 RI-MP2 相关部分的解析静态极化率计算耗时在体系很大时,预期会达到 RI-MP2 相关能计算的 24--30 倍左右,仍然会比数值梯度的计算代价高少许。但一方面,Hartree-Fock 或杂化泛函能量 (SCF) 与极化率 (CP-KS) 计算的耗时在中等体系 (10 重原子左右) 与 RI-MP2 极化率耗时相比经常是相当的;因此对于我们当前工作所关心的分子体系,我比较有把握地认为,解析的 RI-MP2 或双杂化泛函静态极化率确实是效率较高的方法、且比数值梯度效率更高。另一方面,对于其他性质 (譬如频率、NMR、IR 等) 的计算问题,我可能会乐观地认为近乎任何情况下,解析梯度仍然有希望比数值梯度更快。但目前该工作还处于整理阶段,考虑到学业年限问题,这份工作会比较遗憾地无法在这份论文工作中完整呈现。

\subsection*{第 3--5 章工作相关}

从时间上,我的研究生阶段最早开展的工作是第 \alertref{sec.4.title} 章,但最后收尾的工作是第 \alertref{sec.6.title} 章。尽管第 \alertref{sec.4.title} 章并非是第一完成人,也在课题设计的本身并没有深入参与;但在通过对梯度理论 (附录 \alertref{sec.3.title}) 的学习后,第 \alertref{sec.4.title} 章课题的核心主张,即能量与密度相互促进的关系,很大程度上支撑了导师与我自己对性质计算的乐观期待。这也是图 \alertref{fig.6.rel-eng-dm-prop} 表达的想法。在看到第 \alertref{sec.6.title} 章、特别是 \alertref{fig.6.benchmark-compare-xdh-bdh} 相对来说已经很漂亮的结论后,哪怕这样的工作还不确定在将来能否能起到重要的意义,我想所有已经投入在程序实现上的精力,一定算是有了回报。这是确实是非常幸运的:在研究生涯中不成功的尝试会不断地消磨心性,但只要有一些好的结果,就算没有参与这些方法最初的开发,我也感到非常兴奋。但也并非每个人有这样的好运,也并非所有人能耐得住这样的消磨。

图 \alertref{fig.6.rel-eng-dm-prop} 附近的讨论很有价值,即能量、密度与性质相互关系非常密切,且常常性质是密度或其外场扰动的直接导出量。好的泛函如果希望在能量和性质上都有好的表现,其给出的密度是否准确非常关键。但我还是会有些担忧:这样的讨论比起结论,更接近主张、或者不完整的经验总结。尽管密度泛函现在的发展已经愈发 more is different\cite{Anderson-Anderson.S.1972};但作为第一性原理方法,还原论模式仍然非常重要。即使这样的论断很可能没有错误、甚至可能为将来的密度泛函发展提供思路,但如果能有更细致、全面、本质的分析方式,那么这样的论断将不再是主张、而是结论。

同时,现在的计算结果,还无法应用于实验或工况条件。我们所有的计算都只是在逼近所谓的绝热极限;但实验条件还需要考虑振动效应、溶剂效应、实际外场、杂质环境、甚至界面环境与仪器测量误差等等\cite{Mata-Suhm.ACIE.2017, Varandas-Varandas.ARPC.2018}。我的工作环境还是非常干净的,并不考虑理论与实验之间的鸿沟。同时,极化率问题尽管很重要,但通常更关心的是极化率的梯度性质 (Raman 作为三阶性质)、含频极化率 (需要激发态信息) 等进一步衍生的性质。对于非变分的双杂化泛函,依据 Wigner $2n+2$ 规则,三阶性质就需要用到二阶 U 矩阵,解析梯度的理论与实现难度会比二阶梯度更加困难;如果确实有必要发展三阶解析梯度理论,如何导出正确的公式、程序效率如何调优,很可能就需要自动梯度或元编程的技术的辅助。而对于含频极化率,其困难之一是跃迁密度如何合理地获得,以及跃迁密度给出的非含频外场的静态极化率能否可以严格地等于梯度理论下的情形;对于后者,我对严格相等的可能性感到比较悲观,毕竟以我目前所知 MP2 方法也都没有达到这一点。如果要将工作成果应用到实际问题,可能需要走的路还很长。

\section*{非论文工作相关}
\label{sec.finale.acknoledge}

在复旦大学求学的时间,大概是 11 年左右。经历过重度雾霾,经历过疫情的大上海保卫战;经历过山寨手机的没落,经历过互联网应用的发展,正在经历人工智能对语言和艺术的冲击。时间平等地带走每个生命的一部分,平等地给予每个生命同样的机遇,催促着我们不断向前。

对于密度泛函及其相关方法的学习和实践,大概也持续了 9 年左右的时间。这段时间与许多人相遇,相互学习和成长。

首先感谢徐昕教授。在这几年时间里,从课题、知识、经验、治学、入世上,向徐老师学习良多。在入学前,我最初的志向是硕士研究生;但最终可以以直博研究生入学,在这件事上也非常感谢徐老师给予我从事研究的机会,追随老师的背影前进。

我也有幸与许多有志于计算化学或对计算化学有兴趣的人们,有过一段同行与共同生活的时间。在此感谢吴剑鸣博士、金杨阳先生、苏乃强博士、周余伟博士、杨勇硕士、奚晋扬博士、陈征博士、周智硕士、刘章云博士、王鹤博士、丁晨博士、谷永浩博士、孙世超博士、颜文杰博士、李晓松教授、刘弘斌博士、周凝女士、陆丽馨博士、王艺臻女士、杨宇琦博士、蒋明宏先生、郑若昕硕士、张颖教授、申同昊博士、杨露博士、武晴滢硕士、谭诗乾硕士、李亚静硕士、王石嵘先生、方圆博士、翁军辉博士、杨勤杰先生、赖昕铭先生、王湛丰先生、武秉政先生、陈逸诚先生、邬子豪先生、李之韵女士、景驰远先生、朱煜先生、张文昊先生等。同时感谢陈浙宁博士、饶立博士、段赛博士、翁经纬博士、郑晓教授等老师的讨论或学习。

课题之外的工作与生活也受到许多帮助。在此感谢感谢祝锦祥博士、金霖女士在生活上的照顾。感谢参与杨思宇博士、叶天琳硕士的课题交流或合作。感谢杨奕博士、孙冬硕士、强宜澄博士、李峰宇先生、王石嵘先生、杨宇琦博士、蒋明宏先生共同参与学科竞赛。感谢金杨阳先生、茅为嘉博士、龚举文硕士、杨奕博士、叶兆祺博士、康沛林博士在课业、课题与生活上的照顾。感谢丁臻桢先生、樊明凯先生在校生活上的照顾。感谢方彩云博士、曹梦婷硕士、杨慧丽老师、张湛奇博士等辅导员老师的帮助。就读期间进行过运动医学手术,感谢长海医院关节骨病外科团队的治疗;期间行动不便,在此特别感谢金霖女士、祝锦祥博士的照顾,以及杨慧丽老师的帮助。就读期间,我也很有幸拥有电子游戏的兴趣爱好,前面提及的不少同学也是玩伴或者领路人;在此也特别感谢。
