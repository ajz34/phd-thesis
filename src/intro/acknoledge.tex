% !TEX root=./dummy-intro.tex

\begin{acknowledgements}
\label{sec.acknoledge}

本论文工作选题和课题方向指导由导师徐昕教授开展,计算设备与事务工作由吴剑鸣博士指导。第 \alertref{sec.4.title} 章工作发表时,主要完成人是徐昕教授与苏乃强博士;附录 \alertref{sec.3.title} 工作发表时,主要完成人是谷永浩博士。首先对他们表示感谢。课题其他参与、协助或讨论者包括
\begin{itemize}[nosep]
    \item 第 \alertref{sec.2.title} 章:申同昊博士 (MP2/cr 实现讨论),张颖教授 (静态相关讨论);
    \item 第 \alertref{sec.5.title}、\alertref{sec.6.title} 章:颜文杰博士 (绘图讨论),冯儒林博士 (基组收敛性讨论);
    \item 附录 \alertref{sec.3.title}:谷永浩博士、颜文杰博士、金杨阳、苏乃强博士 (梯度理论与程序讨论),\textsc{PySCF} 程序社区 (程序实现框架与讨论),王石嵘 (程序实现讨论),郑若昕硕士、谭诗乾硕士 (RI 近似讨论),诸多其他开源程序社区。
\end{itemize}
本论文作者对包括上述科研工作者在内的所有课题协助者表示感谢,也同时对参与评审的老师的讨论或意见建议表示感谢。

本文已发表工作受到导师参与项目资助,包括国家自然科学基金 21688102, 91427301, 22233002, 21991130、科学挑战专题 TZ2018004、项目 2018YFA0208600, 2021ZD030330 等。其他未录于本文的工作受到导师或合作老师参与项目资助,包括国家自然科学基金 21373053 与第 14 期国家杰出青年科学基金等。本文作者在校获得优秀学业奖学金与学年学业奖学金、以及研究生生活津贴助学金和两学期助教津贴资助,校外获得第五届 Ubiquant Challenge 量化新星挑战赛与第一届“司南杯”量子计算编程挑战赛奖金资助,以及家庭的生活资助。特此对包括上述资助在内的所有资助表示感谢。

其他致谢参阅后记的\hyperref[sec.finale.acknoledge]{非论文工作相关}小节。最后感谢读者的阅读,并期待这份工作能为我、为我们,更深刻地理解自然现象、描绘更美好的未来,而成为帮助我们迈进的基石。

\end{acknowledgements}
