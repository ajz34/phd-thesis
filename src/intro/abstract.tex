% !TEX root=./dummy-intro.tex

\begin{abstract}

密度泛函理论作为电子结构方法,其发展目标是解决原子到分子尺度的微观或固体物质模拟问题。在 Kohn-Sham 理论与程序实现框架下,相对于高精度波函数方法,密度泛函理论及其近似可以通过较低计算代价得到相对准确的结果。因此,密度泛函理论广受欢迎,并大量应用于计算化学、凝聚态物理等学科。

密度泛函近似的核心问题是对交换相关泛函的构建。双杂化泛函作为其中一种构建框架,它引入了严格的交换能与基于 Görling-Levy 二阶微扰的相关能,是 Perdew 等人提出的“Jacob 阶梯”上最高、最接近于真实泛函的框架。近期的众多测评工作表明,双杂化泛函在反应能、分子构型优化、分子振动、核磁共振效应、固体聚合能等问题上,相比于“Jacob 阶梯”上更低的泛函 (如杂化泛函) 有更好的表现。随着 RI、Laplace-Transfrom 等张量近似策略的发展,双杂化泛函能量的计算开销,在数十原子的分子体系下与杂化泛函已较为接近。总地来说,双杂化泛函在理论框架、测评结果、计算效率上都可圈可点。

双杂化泛函框架还有未尽的工作有待完成。本工作系统地对双杂化泛函形式设计、程序实现与测评等若干问题作研究。所面对与尝试解决的问题包括:
\begin{enumerate}[nosep]
\item MP2 型相关能的双杂化泛函在一般主族化学和弱相互作用问题下有很好的表现;但在 HOMO/LUMO gap 较小的体系、以及被认为具有多参考效应的体系下,存在较严重的数值偏差。针对该问题,本工作基于 Görling-Levy 微扰理论并参考波函数理论,在不增加大量计算开销的前提下,于双杂化泛函中引入了成对电子型相关能作为微扰相关能。本工作提出 7 参数泛函框架 XYG6+1,并基于 GMTKN55 集与部分 Minnesota 2015 子集作参数优化。测评结果与参数优化分析表明,成对电子型相关能的引入,确实有助于更好地处理 HOMO/LUMO gap 较小或具有多参考效应的体系,且仍能良好地描述一般主族化学和弱相互作用。但同时,XYG6+1 框架的成对电子型的双杂化泛函对更具有挑战性的分子解离问题的描述还存在一定的困难;该参数模型也容易随被拟合数据的变化而波动,并不非常稳健。我们相信 XYG6+1 框架的泛函向真实泛函迈进了一小步,但仍有很大的提升空间。
\item 相比于 Hartree-Fock 或低阶泛函,准确计算双杂化泛函或 post-HF 性质,通常需要较大的基组;这对计算代价提出挑战。本工作从下述两方面入手:
\begin{itemize}[nosep]
    \item 发展 RI 近似下 MP2 型双杂化泛函的静态极化率程序。该程序可以应对 ωB97X-2 (包含长短程分离交换能) 与 lrc-XYG3 (包含长短程分离相关能) 等众多类型的 MP2 型双杂化泛函,具有灵活的程序设计框架。该程序计算效率不亚于主流的计算化学软件。
    \item 发展静态极化率的 FPA 策略并验证其有效性。该策略合理组合高精度小基组和低精度大基组方法的计算结果,以较小的误差为代价,有效地降低静态极化率计算开销。在此基础上,本工作对 HR46 与 T144 数据集的同性极化率 $\alpha$ 与异性极化率 $\gamma$ 参考值,从 CCSD/aug-cc-pVTZ 或 CCSD(T)/aug-cc-pVTZ 更新到更高精度的 CCSD(T)/CBS 级别。
\end{itemize}
\item 本工作将系统地作如下性质测评,以探究双杂化泛函的表现:
\begin{itemize}[nosep]
    \item 原子体系电子云密度与能量的测评。密度泛函作为电子云密度到基态能量的映射,一个好的近似应当能同时描述好电子云密度与能量两者。Medvedev 等 (\emph{Science}, 355, 49--52) 指出,近期发展的部分密度泛函近似,尽管对基态能量有较好的描述,但对原子体系的电子云密度描述欠佳。Medvedev 等人的工作详尽地测评了“Jacob 阶梯”上低阶泛函的电子云密度,悲观地指出最近发展的部分密度泛函近似并未走在正确的道路上。本工作对 Medvedev 等人的工作进行拓展,测评“Jacob 阶梯”最高阶的双杂化泛函。测评结果表明,双杂化泛函在原子体系的电子云密度上显著优于低阶泛函;同时原子体系能量的表现也相当优异。相比于其他双杂化泛函,xDH 型泛函有更为出色的表现。从原子体系的电子云密度与能量的角度来看,双杂化泛函仍然沿着“Jacob 阶梯”,切实地逼近真实泛函。
    \item 分子体系静态极化率的测评。静态极化率对光学、分子相互作用、Raman 光谱活性的描述至关重要,是计算化学所关心的重要性质。本工作对参考值达到或接近于 CCSD(T)/CBS 的极化率数据集 HR46, T144 作同性极化率 $\alpha$ 与异性极化率 $\gamma$,对数据集 HH101 作同性极化率 $\alpha$ 与极化率分量 $\alpha_{xx}, \alpha_{yy}, \alpha_{zz}$ 的系统测评。测评结果表明,以 XYGJ-OS、XYG6、DSD-PBEPBE-D3BJ 为代表的双杂化泛函极化率误差显著小于低阶泛函的误差。这些双杂化泛函精度不亚于 CCSD 方法,展现出相当高的计算开销与误差精度的性价比。但相比于实验误差精度 0.5\%,双杂化泛函仍然有一定差距。本工作还发现,依据极化率测评表现对双杂化泛函参数进行优化,通常会显现同自旋二阶微扰贡献项较小的趋势。
\end{itemize}
\end{enumerate}

\end{abstract}

\begin{abstract*}

Density functional theory, as an electronic structure method, was developed with the goal of solving the problem of modeling microscopic or solid matter at atomic to molecular scales. Within the framework of Kohn-Sham theory and its program implementation, density functional theory and its approximations can yield relatively accurate results at low computational cost compared to high-precision wavefunction methods. Hence, density functional theory has gained popularity and is heavily used in computational chemistry, condensed matter physics, and other disciplines.

The central problem in density functional approximation is to properly construct exchange-correlation functional. As one of the construction frameworks, doubly hybrid introduces the exact exchange energy, as well as the correlation energy based on the second-order Görling-Levy perturbation. Doubly hybrid functionals lay on the highest-rung of ``Jacob's ladder'' proposed by Perdew et al., closest to the exact functional compared to lower-rung functionals (e.g., hybrid functionals). Numerous recent benchmarks have shown that doubly hybrid functionals perform better than lower-rung functionals in cases such as reaction energies, geometric optimization, molecular vibrations, nuclear magnetic resonance effects, and cohesive energies of condensed-matters. With the development of tensor-decomposition approximation strategies such as resolution-of-identity, Laplace-Transfrom, etc., computational cost of doubly hybrid functional energies is closer to that of lower-rung hybrid functionals for molecular systems with tens of atoms. To summary, doubly hybrid functional is advanced in its theoritical framework, showing considerably well-performed benchmark results with generally low computation cost.

Several research works still to be conducted on doubly hybrid functional framework. This thesis systematically investigates various issues of doubly hybrid functional, including designing, program realization and benchmark. To be more specific,
\begin{enumerate}[nosep]
\item Doubly hybrid functionals with MP2-type correlation perform well in general main-group chemistry and weak interaction problems; however, there are serious numerical discrepancies in systems with small HOMO/LUMO gaps, and in systems that are thought to have multi-reference effects. To address this problem, based on the Görling-Levy perturbation theory, co-opting concepts from wavefunction theory, we introduce electron-pair-type correlation as the perturbation contribution of doubly hybrid functional, without adding considerable computational overhead. In this work, the 7-parameter framework XYG6+1 is proposed. Parameters are optimized with GMTKN55 dataset and some subsets of Minnesota 2015. Benchmark results and the parameter optimization analysis show that, introducing electron-pair-type correlation indeed better describes systems with smaller HOMO/LUMO gaps or with multi-reference effects, while also ensures good performance on general main-group chemistry and weak interaction problems. At the same time, however, the XYG6+1 framework still has difficulties in describing more challenging problems, such as molecular dissociation. The parametric model is also prone to fluctuate with the data being fitted, showing that XYG6+1 framework is not very robust. We believe that the generalization from MP2-type doubly hybrid framework to the XYG6+1 framework is a small step towards the exact density functional, but there is still much room for improvement.
\item Accurate molecular property computation of doubly hybrid functionals or post-HF methods typically require larger basis sets than that of Hartree-Fock or lower-rung functionals. This poses a challenge in terms of computational cost. In this thesis, we start from the following two aspects:
\begin{itemize}[nosep]
    \item Developing static polarizability evaluation program for MP2-type doubly hybrid functionals with resolution-of-identity approximation. The program can cope with many types of MP2-type doubly hybrid functionals, such as ωB97X-2 (including range-separated exchange) and lrc-XYG3 (including range-separated correlation), and has a flexible programming framework. Computational efficiency of this program is comparable to that of mainstream computational chemistry softwares.
    \item Developing and validating FPA strategy for static polarizability. This strategy properly combines results of more demanding elec\-tronic-structure methods with small basis set, and coarser electronic-struc\-ture methods with small basis set, successfully reducing computation overhead at the expense of relatively small error. Based on this FPA strategy, this work improves isotropic polarizability $\alpha$ and anisotropic polarizability $\gamma$ reference data of HR46 and T144 from CCSD/aug-cc-pVTZ or CCSD(T)/aug-cc-pVTZ to more accurate CCSD(T)/CBS.
\end{itemize}
\item In this thesis, following systematical benchmarks of molecular properties will be conducted to explore the performance of doubly hybrid functionals:
\begin{itemize}[nosep]
    \item Benchmark of electronic density and energy of atomic systems. Medvedev \textit{et al.} (\emph{Science}, 355, 49--52) pointed out that, while some of the recently developed density functional approximations can accurately describe energy of the ground state, they do not produce the correct electronic density of atomic systems. The work of Medvedev \textit{et al.} thoroughly benchmarks electronic density of low-rung functionals on the ``Jacob's ladder'', pessimistically suggesting that the some of the recently developed density functional approximations are straying from the path toward the exact functional. The present work extends the work of Medvedev \textit{et al.} by benchmarking doubly hybrid functionals, as functionals laying on the highest rung on the ``Jacob's ladder''. Benchmark results show that the doubly hybrid functional is significantly better than the lower-rung functionals in terms of electronic density of the atomic system, and the energy of the atomic system is also quite well described at the same time. Compared with other doubly hybridized functionals, the xDH-type functionals gain better benchmark results. From this perspective, doubly hybrid functionals are still climbing the path toward the exact functional along the ``Jacob's ladder''.
    \item Benchmark of static polarizabilities of molecular systems. Static polarizabilities are crucial for description of optics, molecular interactions, Raman spectroscopic activity, and are important properties of interest to computational chemistry. In this work, datasets with reference value precision at or close to CCSD(T)/CBS are considered. Thorough benchmark are conducted to isotropic polarizabilities $\alpha$ and anisotropic polarizabilities $\gamma$ of HR46 and T144 datasets, as well isotropic polarizabilities $\alpha$ and polarizability components $\alpha_{xx}, \alpha_{yy}, \alpha_{zz}$ of HH101 dataset. Benchmark results show that polarizability errors of doubly hybrid functionals represented by XYGJ-OS, XYG6 and DSD-PBEPBE-D3BJ are significantly smaller than those of the low-rung functionals. The accuracy of these doubly hybrid functionals are comparable or even better than that of CCSD, demonstrating considerable cost-effectiveness in terms of computational overhead and accuracy. However, compared with the experimental error accuracy of 0.5\%, the doubly hybrid functionals still slightly fall short. It is also found that optimizing the parameters of the doubly hybrid functional based on the performance of the polarizability benchmarks usually shows a tendency for the same-spin contribution of second-order perturbation to be smaller.
\end{itemize}
\end{enumerate}
    
\end{abstract*}
