% !TEX root=./dummy-intro.tex

\begin{abstract}

密度泛函理论作为电子结构方法,其发展目标是解决原子到分子尺度的微观或固体物质模拟问题。在 Kohn-Sham 理论与程序实现框架下,相对于高精度波函数方法,密度泛函理论及其近似可以通过较低计算代价得到相对准确的结果;因此,它广泛应用于理论与计算化学、凝聚态物理、计算材料等学科。

密度泛函近似的核心问题是对交换相关能泛函的构建。双杂化泛函作为交换相关能泛函的构建框架,引入了严格的交换能与基于 Görling-Levy 二阶微扰的相关能;是 Perdew 等人提出“Jacob 阶梯”概念上最高、最严格、最接近于真实泛函的框架。近期的众多测评工作表明,双杂化泛函在反应能、分子构型优化、分子振动、核磁共振效应、固体聚合能等问题上,相比于“Jacob 阶梯”上更低的泛函 (如杂化泛函) 有良好的表现。随着 RI、Laplace-Transfrom 等张量近似策略的发展,双杂化泛函能量的计算开销,在数十原子的分子体系下与杂化泛函已经较为接近。因此,双杂化泛函在理论框架、测评结果、计算效率上都有可观的表现。

双杂化泛函框架还有未尽的工作有待完成。本工作系统地对双杂化泛函设计、程序实现与测评的若干问题作研究。所面对与尝试解决的问题包括:
\begin{enumerate}[nosep]
\item MP2 型相关能的双杂化泛函在一般主族化学和弱相互作用问题下有很好的表现;但在 HOMO/LUMO gap 较小的体系、以及被认为具有多参考效应的体系下,存在较严重的数值偏差。针对该问题,本工作基于 Görling-Levy 微扰理论并参考波函数理论,在不增加大量额外计算开销的前提下,于双杂化泛函中引入了成对电子型相关能作为微扰相关能。本工作提出 7 参数泛函框架 XYG6+1,并基于 GMTKN55 数据集与部分 Minnesota 2015 数据子集作参数优化。测评结果以及细致的参数优化分析表明,成对电子型相关能的引入,确实有助于更好地处理 HOMO/LUMO gap 较小或具有多参考效应的体系,且仍能保证良好地处理一般主族化学和弱相互作用问题。但同时,XYG6+1 框架的成对电子型双杂化泛函对更具有挑战性的分子解离问题的描述还存在困难;该参数模型也容易随被拟合数据的变化而波动,并不非常稳健。我们相信 XYG6+1 框架的泛函向真实泛函迈进一小步,但仍有很大的提升空间。
\item 准确计算 MP2 型双杂化泛函或 post-HF 方法性质,通常需要较大的基组;这对计算代价提出挑战。针对电性质、特别是静态极化率计算问题,本工作从下述两方面入手:
\begin{itemize}[nosep]
    \item 发展 RI 近似下的 MP2 型双杂化泛函的静态极化率程序。该程序可以应对 ωB97X-2 (包含长短程分离交换能) 与 lrc-XYG3 (包含长短程分离相关能) 等众多类型的 MP2 型双杂化泛函,具有灵活的程序设计框架。该程序计算效率不亚于主流的计算化学软件。
    \item 发展 FPA 策略并验证其有效性。该策略合理地组合高精度小基组、和低精度大基组方法的计算结果,以较小的误差为代价,成功地降低静态极化率计算开销。在此基础上,本工作对极化率数据集 HR46 与 T144 的同性极化率 $\alpha$ 与异性极化率 $\gamma$ 参考值,从 CCSD/aug-cc-pVTZ 或 CCSD(T)/aug-cc-pVTZ  更新到更高精度的 CCSD(T)/CBS 级别。
\end{itemize}
\item 本工作将系统地作如下性质测评,以探究双杂化泛函的表现:
\begin{itemize}[nosep]
    \item 原子体系电子云密度与能量的测评。密度泛函作为电子云密度到基态能量的映射,一个好的近似应当能同时描述好电子云密度与能量两者;但 Medvedev 等 (\emph{Science}, 355, 49--52) 指出,近期发展的部分密度泛函近似,尽管对基态能量有较好的描述,但对原子体系的电子云密度描述欠佳。Medvedev 等人的工作详尽地测评了“Jacob 阶梯”上低阶泛函 (不高于杂化泛函) 原子体系的电子云密度,悲观地指出最近发展的密度泛函近似并未走在正确的道路上。本工作对 Medvedev 等人的工作进行拓展,测评“Jacob 阶梯”最高阶的双杂化泛函。结果表明,双杂化泛函在原子体系的电子云密度上显著优于低阶泛函;同时原子体系能量的表现也相当优异。相比于其他双杂化泛函,xDH 型泛函有更为出色的表现。这表明,从原子体系的电子云密度与能量的角度来看,双杂化泛函仍然沿着“Jacob 阶梯”,切实地逼近真实泛函。
    \item 分子体系静态极化率的测评。静态极化率可以对光学、分子相互作用、Raman 光谱活性的描述至关重要,是计算化学所关心的重要性质。本工作对具有参考值达到或接近于 CCSD(T)/CBS 的极化率数据集 HR46, T144 作同性极化率 $\alpha$ 与异性极化率 $\gamma$,对数据集 HH101 中自旋极化与自旋非极化的体系分别作同性极化率 $\alpha$ 与极化率分量 $\alpha_{xx}, \alpha_{yy}, \alpha_{zz}$ 的系统测评。测评结果表明,以 XYGJ-OS、XYG6、DSD-PBEPBE-D3BJ 为代表的双杂化泛函极化率误差显著小于低阶泛函的误差,且精度不亚于 CCSD 方法,展现出相当高的计算开销与误差精度的性价比;但相比于实验误差精度 0.5\% 仍然有一定差距。本工作还发现,依据极化率测评表现对双杂化泛函参数进行优化,通常会显现同自旋二阶微扰贡献项较小的趋势。
\end{itemize}
\end{enumerate}

\end{abstract}