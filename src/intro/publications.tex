% !TEX root=./dummy-intro.tex

\chapter{攻读学位期间研究成果}

% 盲审期间显示发表大体情况

% \subsection*{论文相关工作}

% 论文影响因子 (IF) 由 \href{https://jcr.clarivate.com/jcr/browse-journals}{Journal Citation Reports\textsuperscript{TM}} 服务提供。查阅时间为 2024 年 3 月 16 日。下述 IF 展示的是 2017--2022 年的五年影响因子。作者名单中,粗体表示学位申请者、下划线表示通讯作者。

% \begin{itemize}[nosep]
%     \item 第 \alertref{sec.2.title} 章工作:
%     \begin{itemize}[nosep]
%         \item 待发表工作;
%     \end{itemize}
%     \item 第 \alertref{sec.4.title} 章工作:
%     \begin{itemize}[nosep]
%         \item Su, N. Q.; \textbf{Zhu, Z.}; \underline{Xu, X.} Doubly Hybrid Density Functionals That Correctly Describe Both Density and Energy for Atoms. \textit{Proc. Natl. Acad. Sci.} \textbf{2018}, \textit{115}, 2287–2292. \\
%         doi: \href{https://doi.org/10.1073/pnas.1713047115}{10.1073/pnas.1713047115}
%         (IF: 12.0)
%         \item \textbf{祝震予},苏乃强,\underline{徐昕}。双杂化泛函可以更好地同时描述密度与能量性质 (墙报展示)。中国化学会第 31 届学术年会,2018。
%     \end{itemize}
%     \item 第 \alertref{sec.5.title} 章工作:
%     \begin{itemize}[nosep]
%         \item \textbf{Zhu, Z.}; \underline{Xu, X.} Focal-Point Analysis to Achieve Accurate CCSD(T) Data Set References for Static Polarizabilities. \textit{J. Chem. Theory Comput.} \textbf{2023}, \textit{19}, 3112–3122.\\
%         \href{https://doi.org/10.1021/acs.jctc.3c00025}{10.1021/acs.jctc.3c00025}
%         (IF: 5.8)
%     \end{itemize}
%     \item 第 \alertref{sec.6.title} 章工作:
%     \begin{itemize}[nosep]
%         \item 待发表工作;
%         \item \textbf{祝震予},\underline{徐昕}。 xDH 双杂化泛函静态极化率实现与测评.第 14 届全国量子化学会议,2021。
%     \end{itemize}
%     \item 附录 \alertref{sec.3.title} 工作:
%     \begin{itemize}[nosep]
%         \item Gu, Y.; \textbf{Zhu, Z.}; \underline{Xu, X.} Second-Order Analytic Derivatives for XYG3 Type of Doubly Hybrid Density Functionals: Theory, Implementation, and Application to Harmonic and Anharmonic Vibrational Frequency Calculations. \textit{J. Chem. Theory Comput.} \textbf{2021}, \textit{17}, 4860–4871. \\
%         doi: \href{https://doi.org/10.1021/acs.jctc.1c00457}{10.1021/acs.jctc.1c00457}
%         (IF: 5.8)
%     \end{itemize}
% \end{itemize}

% \subsection*{其他工作}

% \begin{itemize}[nosep]
%     \item Shen, T.; \textbf{Zhu, Z.}; \underline{Zhang, I. Y.}; Scheffler, M. Massive-Parallel Implementation of the Resolution-of-Identity Coupled-Cluster Approaches in the Numeric Atom-Centered Orbital Framework for Molecular Systems. \textit{J. Chem. Theory Comput.} \textbf{2019}, \textit{15}, 4721–4734. \\
%     doi: \href{https://doi.org/10.1021/acs.jctc.8b01294}{10.1021/acs.jctc.8b01294}
%     (IF: 5.8);
%     \item Ye, T.; Huang, Z.; \textbf{Zhu, Z.}; Deng, D.; Zhang, R.; \underline{Chen, H.}; \underline{Kong, J.} Surface-Enhanced Raman Scattering Detection of Dibenzothiophene and Its Derivatives without π Acceptor Compound Using Multilayer Ag NPs Modified Glass Fiber Paper. \textit{Talanta} \textbf{2020}, \textit{220}, 121357. \\
%     doi: \href{https://doi.org/10.1016/j.talanta.2020.121357}{10.1016/j.talanta.2020.121357}
%     (IF: 5.4)
%     \item Wu Q.; \textbf{Zhu Z.}; \underline{Wu J.}; Xu X. A Dataset Representativeness Metric and A Slicing Sampling Strategy for the Kennard-Stone Algorithm. \textit{Chem. J. Chin. Univ.} \textbf{2022}, \textit{43}, 20220397. \\
%     \href{https://doi.org/10.7503/cjcu20220397}{10.7503/cjcu20220397}
%     (IF: 0.6)
% \end{itemize}

\subsection*{论文相关工作}

论文影响因子 (IF) 由 \href{https://jcr.clarivate.com/jcr/browse-journals}{Journal Citation Reports\textsuperscript{TM}} 服务提供。查阅时间为 2024 年 3 月 16 日。下述 IF 展示的是 2017--2022 年的五年影响因子。

\begin{itemize}[nosep]
    \item 第 \alertref{sec.2.title} 章工作:
    \begin{itemize}[nosep]
        \item 待发表工作;
    \end{itemize}
    \item 第 \alertref{sec.4.title} 章工作:
    \begin{itemize}[nosep]
        \item \emph{Proc. Natl. Acad. Sci.} (IF: 12.0),第二作者;%10.1073/pnas.1713047115
        \item 中国化学会学术年会;
    \end{itemize}
    \item 第 \alertref{sec.5.title} 章工作:
    \begin{itemize}[nosep]
        \item \emph{J. Chem. Theory Comput.} (IF: 5.8),第一作者;%10.1021/acs.jctc.3c00025
    \end{itemize}
    \item 第 \alertref{sec.6.title} 章工作:
    \begin{itemize}[nosep]
        \item 待发表工作;
        \item 全国量子化学会议;
    \end{itemize}
    \item 附录 \alertref{sec.3.title} 工作:
    \begin{itemize}[nosep]
        \item \emph{J. Chem. Theory Comput.} (IF: 5.8),第二作者。%10.1021/acs.jctc.1c00457
    \end{itemize}
\end{itemize}

\subsection*{其他工作}

\begin{itemize}[nosep]
    \item \emph{J. Chem. Theory Comput.} (IF: 5.8),第二作者;%10.1021/acs.jctc.8b01294
    \item \emph{Talanta} (IF: 5.4),第三作者;%10.1016/j.talanta.2020.121357
    \item 高等学校化学学报 (IF: 0.6),第二作者。%10.7503/cjcu20220397
\end{itemize}


