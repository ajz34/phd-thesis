%-----全局定义-----
\documentclass[type=doctor]{fduthesis}
% \usepackage{fdudoc}

%-----FDU thesis setup-----
\fdusetup{
    style = {
        font = times,
        cjk-font = fandol,
        font-size = -4,
        fullwidth-stop = mapping,
        % footnote-style = xits,
        hyperlink = color,
        hyperlink-color = default,
        bib-backend = bibtex,
        bib-resource = {../thesis.bib},
        % bib-style = achemso,
        % cite-style = numerical,
        % declaration-page = {declaration.pdf},
        % 插入扫描版的声明页 PDF 文档
        % 默认使用预定义的声明页,但不带签名
        auto-make-cover = false
        % 是否自动生成论文封面(封一)、指导小组成员名单(封二)和声明页(封三)
        % 除非特殊需要(e.g. 不要封面),否则不建议设为 false
    },
    %
    % info 类用于录入论文信息
    info = {
    title = {双杂化密度泛函分子能量与性质\\计算方法的进展与测评},
    title* = {\alert{Thesis Title}},
    % 英文标题
    %
    author = {祝震予},
    supervisor = {徐\quad 昕\quad 教授},
    major = {物理化学},
    degree = academic,
    department = {化学系},
    student-id = {17110220038},
    % date = {2023 年 1 月 1 日},
    % 日期
    % 注释掉表示使用编译日期
    instructors = {
        {徐\quad 昕 \quad 教\quad 授},
    },
    % 指导小组成员
    % 使用英文逗号 “,” 分隔
    % 如有需要,可以用 \quad 手工对齐
    %
    keywords = {\alert{不确定关系, 量子力学, 理论物理}},
    % 中文关键词
    % 使用英文逗号 “,” 分隔
    %
    keywords* = {Uncertainty principle, quantum mechanics, theoretical physics},
    % 英文关键词
    % 使用英文逗号 “,” 分隔
    %
    clc = {O641.12},
    % 中图分类号
    }
}

%-----fduthesis issues-----
% issue #86
\ExplSyntaxOn
\tl_set:Nn \c__fdu_cover_info_align_tl { c @ { \c__fdu_fwid_colon_tl } l }
\ExplSyntaxOff
% 化学系图表格式要求
\ExplSyntaxOn
\cs_set:Npn \thefigure
{ \thechapter . \__fdu_arabic:n { figure } }
\cs_set:Npn \thetable
{ \thechapter . \__fdu_arabic:n { table } }
\ExplSyntaxOff

%-----图表设置-----
\usepackage{siunitx}
\usepackage{enumitem}
\newcommand{\tabnote}[1]{\textsuperscript{\emph{#1}}}
\usepackage{threeparttable}
\usepackage{threeparttablex}
\usepackage{graphicx}
\usepackage{longtable}
\usepackage{longfigure}
\usepackage{subcaption}
\usepackage{float}
\usepackage{lscape}
\usepackage{multicol}
\usepackage{multirow}
\usepackage{arydshln}
\usepackage{dcolumn}
\newcolumntype{d}[1]{D{.}{.}{#1}}
\setlength\dashlinedash{0.5pt}
\setlength\dashlinegap{1.5pt}
\setlength\arrayrulewidth{0.5pt}
\usepackage[figuresright]{rotating}
% \usepackage{booktabs}
\usepackage{tabularray}
\UseTblrLibrary{booktabs}

%-----化学符号-----
\usepackage[version=4]{mhchem}

%-----数学记号----
\usepackage[ntheorem]{empheq}
\allowdisplaybreaks[1]

%-----其它定义-----
\usepackage{ifthen}
\newcommand{\Schrodinger}{Schr\"o\-dinger}

% tablenotes 与表格内注释超链接 (from fdudoc.cls)
\makeatletter
\renewlist{tablenotes}{description}{1}
\setlist[tablenotes]{
  format      = \normalfont\itshape\tnote@item,
  labelwidth  = 0.5em,
  itemindent  = 0pt,
  rightmargin = \tabcolsep,
  leftmargin  = \the\dimexpr\tabcolsep+1em\relax,
  after       = \@noparlisttrue}
\AtBeginEnvironment{tablenotes}{%
  \setlength\parindent{2\ccwd}%
  \normalfont\footnotesize}
\AtBeginEnvironment{threeparttable}{%
  \stepcounter{tpt@id}%
  \edef\curr@tpt@id{tpt@\arabic{tpt@id}}}
\newcounter{tpt@id}
\def\tnote@item#1{%
  \Hy@raisedlink{\hyper@anchor{\curr@tpt@id-#1}}#1}
\def\TPTtagStyle#1{\textit{\hyperlink{\curr@tpt@id-#1}{#1}}}
\makeatother

% 用于表格注释与 threeparttable 环境引入的便利函数
\renewcommand{\TPTminimum}{\linewidth}
\newcommand{\widetabular}[2]{%
\ifx&#2&
  \begin{threeparttable}
    \centerline{\makebox[2\linewidth]{#1}}
  \end{threeparttable}
\else
  \begin{threeparttable}
    \centerline{\makebox[2\linewidth]{#1}}
  \begin{tablenotes}[nosep, topsep=0.5em]
    #2
  \end{tablenotes}
  \end{threeparttable}
\fi}

% 用于分章节编译与统稿的代码
\newcommand{\alert}[1]{{\color{red}{#1}}}
\newcommand{\alertref}[1]{{\color{red}{#1}}}

% 用于表示方法的格式
\newcommand{\textmt}[1]{\textsf{#1}}

% 向量加粗的简记
\newcommand{\bm}{\symbfit}

% 保证 mathbb 被花括号包含
\renewcommand{\mathbb}[1]{{\symbb{#1}}}

%---------设定区结束----------

% 格式检查列表
% [ ] 表格数据使用 \widetabular{}{} 插入,以替代自定义的 \tabnote 和默认的 threeparttable。
% [ ] 表格 caption 在上,图片 caption 在下。图片不引入注释。
% [ ] 表格尽可能不引入纵向分割线。
% [ ] 建构术语表与符号表,避免文中出现术语定义、特别是英文定义。
